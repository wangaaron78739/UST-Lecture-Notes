\documentclass[../main/main.tex]{subfiles}


\begin{document}

\section{September  4th, 2019}
\subsection{Introduction to NLP}
\begin{description}
	\item[Reference Material] 
	\item CS 224N (Chris Manning)
	\item ICML tutorial on Natural Language Underanding\ldots (Percy Liang)
\end{description}
Here are some common tasks in NLP:
\begin{description}
	\item[Morphology]: basic unit of words, e.g. 
	\item[Syntax]: what is grammatical? 
	\item[Semantics]: What does it meam?
	\item[Pragmatics]
\end{description}
\begin{example}
	\[
	2+3\iff 3+2
	.\] Different syntax, same semantics
\end{example}
\begin{example}
	\[
		\text{3/2} \text{ (Python 2.7)} \iff \text{3/2} \text{ (Python 3)}
	.\] Same syntax different semantics
\end{example}
\begin{example}
	[Parts of Speech] TODO::
\end{example}
\end{document}

