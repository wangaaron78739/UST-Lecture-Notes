\documentclass[../main/main.tex]{subfiles}


\begin{document}

\section{September  6th, 2019}
\subsection{Language Models}
\begin{definition}
	A \vocab{statistical language model} is a model specifying probability distribution over word sequences. It can be regarded as a probabilistic mechanism for ``generating'' text, thus it's also called a ``generative'' model.
\end{definition}
\begin{remark}
	In other words, if you have a language model, you can generate text.
\end{remark}
LM's are useful as they provide a principled way to quantify the uncertainties associated with natural language, helping us with \vocab{speech recognition}, \vocab{text categorization}, and \vocab{information retrieval}.\\

We can also use conditional probability, e.g. given we've seen the word ``basketball'', it's likely we're talking about sports.\\

\subsection{Review On Probability}

\[
	P(x|y)=\frac{P(x,y)}{P(y)}=\frac{P(y|x)P(x)}{P(y)}
.\] 
Thus,
\[
	\max_x P(x|y)=\max_x P(y|x)P(x)
.\] 
\end{document}

