\documentclass[../main/main.tex]{subfiles}


\begin{document}

\section{September  2nd, 2019}
\subsection{Association Rule Mining}
Suppose we have the following dataset:
\begin{table}[htpb]
	\centering
	%\caption{caption}
	\label{tab:label}
	\begin{tabular}{|c|ccc|}
	\hline
	Customer && Shopping List &\\
	\hline
	Raymond & Apple & Coke & Coffee \\
	David & Diaper & Coke & \\
	Emily & Milk & Biscuit & \\
	Derek & Coke & Milk & \\
	\hline
	\end{tabular}
\end{table}\\
\begin{definition}
	The things on the RHS are \vocab{items}.
\end{definition}
\begin{definition}
	For each customer we have their \vocab{history} or \vocab{transaction}.
\end{definition}
We want to find some \vocab{associations} between items. An example of an interesting association might be:
\begin{example}[Example of an interesting association]
	{Diapers} and {Beers} are usually bought together.
\end{example}
This association could have different reasons, e.g. people buy both diapers and beer after work usually.
\subsection{Applications of Association Rule Mining}
Here are some examples of where association rule mining might be used:
\begin{itemize}
	\item Supermarket - For recommendation
	\item Web Mining - Google for their autocomplete
	\item Medical Analysis - Diagnosis from the patient's attributes or finding key attributes linked to illnesses (diabetes and obesity)
	\item Bioinformatics - Patterns in genomes
	\item Network Analysis - Associating IP and DoS, e.g. seeing if your packet goes through
	\item Programming Patern Finding - e.g. linking segmentation faults and users
\end{itemize}
\subsection{Problem Definition}
Consider the following dataset (TID = Transaction ID):
\begin{itemize}
	\item TID: $t_1$, Items: $A$, $D$
	\item TID: $t_2$, Items: $A$, $B$, $D$, $E$
	\item TID: $t_3$, Items: $B$, $C$
	\item TID: $t_4$, Items: $A$, $B$, $C$, $D$, $E$
	\item TID: $t_5$, Items: $B$, $C$, $E$
\end{itemize}
In table form this would be:
\begin{table}[htpb]
	\centering
	%\caption{caption}
	\label{tab:label}
	\begin{tabular}{|c|c|c|c|c|c|}
	\hline
	TID & $A$ & $B$ & $C$ & $D$ & $E$ \\
	\hline
	$t_1$ & 1 & 0 & 0 & 1 & 0 \\
	$t_2$ & 1 & 1 & 0 & 1 & 1 \\
	$t_3$ & 0 & 1 & 1 & 0 & 0 \\
	$t_4$ & 1 & 1 & 1 & 1 & 1 \\
	$t_5$ & 0 & 1 & 1 & 0 & 1 \\
	\hline
	\end{tabular}
\end{table}
\begin{definition}
	A \vocab{single item} is a single item (duh).
\end{definition}
\begin{example}[Examples of Single Items]
	$A, B, C, D,$ or $E$
\end{example}
\begin{definition}
	An \vocab{itemset} is a set of items (again, duh).
\end{definition}
\begin{example}[Examples of Itemsets]\label{itemset_ex}
	$\{B,C\}, \{A,B,C\}, \{B,C,D\}, \{A\}$
\end{example}
\begin{definition}
	An \vocab{$n$-itemset} is a set of $n$ items.
\end{definition}
\begin{example}
	[Examples of $n$-itemsets]
	From Example \ref{itemset_ex}, we have the following:
	\begin{itemize}
		\item $\{B,C\} $ is a $2$-itemset
		\item $\{A,B,C\} $ and $\{B,C,D\} $ are $3$-itemsets
		\item $\{A\} $ is a $1$-itemset
	\end{itemize}
\end{example} 
\begin{definition}
	The \vocab{support} (or \vocab{frequency}) of an item or an itemset is the number of times it appears in the dataset.
\end{definition}
\begin{example}
	[Examples of the Support of Items and Itemsets]
	From Example \ref{itemset_ex}, we have the following:
	\begin{itemize}
		\item The support of $A$ is 3:  $A \in t_1,t_2,t_4$
		\item The support of $B$ is 4: $B \in t_2,t_3,t_4,t_5$
		\item The support of $\{B,C\} $ is 3: $\{B,C\} \subseteq t_3,t_4,t_5$
		\item The support of $\{A,B,C\} $ is 3: $\{A,B,C\} \subseteq t_4$
	\end{itemize}
\end{example}
As such, we might try to classify large itemsets or \vocab{frequent itemsets} as itemsets with support greater than a threshold, e.g. 3.
\begin{definition}
	An $n$-frequent itemset is an itemset with support $n$.
\end{definition}
\begin{example}
	$\{B,C\} $ is a $3$-frequent itemset of size $2$.
\end{example}
\begin{definition}
	An \vocab{association rule} is a association between an item/itemset and another.
\end{definition}
\begin{definition}
	The \vocab{support} of an association rule is the number of transaction with both the LHS and RHS of the association rule.
\end{definition}
\begin{definition}
	The \vocab{confidence} of an association rule is the support of the association rule divided by the number of transaction with the LHS of the rule.
\end{definition}
\begin{example}
	$\{B,C\} \to E$ is an example of an association rule. It has a support of 3 ($t_3,t_4,t_5$) and a confidence of $\frac{2}{3}$ (it's true for $t_3$ and $t_4$ but not $t_5$).
\end{example}
In essence, we want to find association rules with:
\begin{itemize}
	\item Support greater than a threshold e.g. ($\ge 3$)
	\item Confidence greater than a threshold e.g. ($\ge 50\%$)
\end{itemize}
We can do split this into two steps:
\begin{enumerate}
	\item Find all ``large" itemsets (e.g. itemsets with support $\ge 3$)
	\item Find all ``interesting" association rules after Step 1:
		\begin{itemize}
			\item From all ``large" itemsets, find the association rules with confidence $\ge 50\%$.
			\item This can be done by taking every pair of elements from Step 1, $X$ and  $Y$, where  $X\subset Y$, and checking if $\frac{\text{supp}(Y)}{\text{supp}(X)}\ge 50\%$. 
			\item If yes, then generate the rule: ``$X\to Y-X$"
			\end{itemize}
\end{enumerate}
\begin{description}
	\item[Homework]
	\item Show that the support of the association rule is still large. This can be easily seen, as $X$ is large, and  $Y-X\subseteq Y$, making it large from \ref{largesubsets}
\end{description}
\end{document}



