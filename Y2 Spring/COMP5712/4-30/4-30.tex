\documentclass[../main/main.tex]{subfiles}


\begin{document}

\section{April 30th, 2019}
\subsection{Randomized Multiway Cut Problem}
\begin{figure}[h!]
	\centering
	\includegraphics[width=0.4\textwidth]{4-30-rand}
	\caption{For $k=3$}
	\label{fig:4-30-rand}
\end{figure}
\begin{itemize}
	\item Select $r\in (0,1)$ uniformly at random
	\item Consider ``corners" of simplex in random order
	\item Each corner considered ``grabs" all vertices within ``ball-distance''  $r$ from it that have not yet been grabbed. i.e. \[
			u\in B(S,r)
	.\] 
	\item The last corner considered grabs all remaining vertices.
	\item To achieve an approximation factor of $1.5$, we need to show that the probability that $u$ and $v$ are assigned to different corners is $\le \frac{3}{4 \| x_u-x_v\|}$.
\end{itemize}

\begin{lemma}
	$u\in B(S_i,r)$ if and only if $1-x^i_u\le r$.
\end{lemma}
\begin{proof}
	Ball distance of $x_u$ from $S_i = \frac{1}{2}\|x_u-S_i\|$ \[
		=\frac{1}{2}\left[ (1-x_u^i)+ \sum_{j\neq i} x_u^j \right]=\frac{1}{2}\left[ (1-x^i_u)+(1-x^i_u \right] =1-x^i_u
	.\] This is because the $j$-th component $S_i$ is $0$ for all $j\neq i$ and $1$ for $j=i$, and $\sum x^j_u=1$.
\end{proof}

\begin{definition}
	We say that an index $i$ \vocab{cuts} $(u,v)$ if \underline{exactly} one of $x_u$ and $x_v$ is contained in ball $B(S_i,r)$.
\end{definition}

For $x_u$ and $x_v$ to be assigned in different corner, it if necessary that there is some index that cuts  $(u,v)$. This is not sufficient since $u$ and $v$ could both be grabbed by a vertex, but another vertex later on might cut the edge (since they are selected in a random order).\\

When does index $i$ cut $(u,v)?$ In fact, what is the probability that index  $i$ cuts $(u,v).$ From the lemma above, we only need to look at that $i$-th coordinate of $x_u^i$. \\

WLOG, assume that $x_u$ is closer to $S_i$. Then \[\text{$(u,v)$ is cut by index $i$ iff  $r\in (1-x^i_u,1-x^i_v)$}.\] \[
	\implies \text{Pr(index $i$ cuts $(u,v)$)}= \frac{(1-x^v_i)-(1-x^u_i)}{1}=|x^i_u-x^i_v|
.\] \\
Now we have: \[
	\text{Pr($(u,v)\in $ multicut)}= \text{Pr($u $ \& $  v$ are cut by index  $i$)}\] \[
	\le \sum_i \text{Pr($u $ \& $ v$ are cut by index $i$ )}
	=\sum_i |x^i_u-x^i_v|= \|x_u-x_v\|.\] 
Now we would like to tighten the restriction to achieve an upper bound of $\frac{3}{4}\|x_u-x_v\|$. Note that we have not yet used the fact that ``corners'' are considered in random order.

\begin{definition}
	We day that an index $i$ \vocab{settles} $(u,v)$ if $i$ is the first index in the random permutation such that at least one of $u$ or $v$ belongs to $B(S_i,r)$.
\end{definition} 

Intuitively, $i$ decides whether $(u,v)$ will be in the multiway cut or not. Let us define the following events:
\begin{itemize}
	\item $S_i$ : Event that index  $i$ settles $(u,v)$.
	\item $X_i$ : Event that index $i$ settles $(u,v)$.
\end{itemize}
Then: \[
	\text{Pr}\left[ (u,v)\in \text{ multicut} \right] \le \sum_i \text{Pr}(S_i \wedge X_i)
.\] This is because $(u,v)$ is in the multiway cut only if there is some index $i$ that both settles and cuts $(u,v)$, i.e. it is the first index that separates $u$ and $v$. This is clearly necessary, but it is not sufficient, since it could be the last index.

Let $l$ be the index such that $S_i$ is closest to one of the two endpoints of $(u,v)$. That is, $i=l$ minimizes the quantity $\min\left(\|S_i-u\|,\|S_i-v\|\right)$

\begin{claim}
	Index $i\neq l$ cannot settle edge $(u,v)$ if $l$ is ordered before $i$ in the random permutation.
\end{claim}
\begin{proof}
	If index $i$  were to settle $(u,v)$, then it would be $l$.
\end{proof}
We need to compute $\text{Pr}(S_i\wedge X_i)$ for index $i$. There are 2 cases: $i\neq l$ and $i=l$.
 \begin{enumerate}
	\item For $i\neq l$, we have:  \[
			\text{Pr}\left[ S_i\wedge X_i \right]\]\[ =\text{Pr}\left[ S_i\wedge X_i\ |\ l \text{ occurs after $i$ in permutation} \right] \times \text{Pr}[ l \text{ occurs after $i$ in permutation]} \]\[+ \text{Pr}\left[ S_i \wedge X_i\ |\ l\text{ occurs before $i$ in permutation} \right] \times \text{Pr}\left[ l \text{ occurs before $i$ in permutation} \right] 
			\] \[=\frac{1}{2}\text{Pr}[S_i \wedge X_i\ |\ l\text{ occurs after $i$ in permutation}] \le \frac{1}{2}\text{Pr}[X_i] = \frac{1}{2}|x_u^i-x_v^i|
	.\] 
\item For $i=l$ we have:  \[
		\text{Pr}[S_l \wedge X_l]\le  \text{Pr}[X_l]=|x_u^l-x_v^l|
.\]
\end{enumerate}
This means that: \[
	\text{Pr}\left[ (u,v)\in \text{ multicut} \right] \le \sum_i \text{Pr}(S_i \wedge X_i) \le 
	\frac{1}{2}\sum_{i\neq l} |x^i_u-x^i_v|+|x_u^l-x_v^l|
.\] 
\begin{lemma}
	For any index $l$ and any $2$ vertices $u$ and $v$, \[
	|x^l_u-x_v^l|\le \frac{1}{2}\|x_u-x_v\|
	.\] 
\end{lemma}
\begin{proof}
	$|x_u^l-x_v^l|=\left| \left(1-\sum\limits_{j\neq l} x^j_u\right) - \left( 1-\sum\limits_{j\neq l} x_v^l \right)  \right|=\left| \sum\limits_{j\neq l} \left( x_v^j-x_u^j \right)  \right| \le \sum\limits_{i\neq j} |x_v^j-x_u^j|+\|x_v^l-x_u^l\| $
\end{proof}
\[
\text{Pr}\left[ (u,v)\in \text{ multicut} \right]  \le 
\frac{1}{2}\sum_{i} |x^i_u-x^i_v|+ \frac{1}{2}|x_u^l-x_v^l|\le \frac{1}{2}\|x_u-x_v\|+\frac{1}{2}\times \frac{1}{2} \|x_u-x_v\|=\frac{3}{4}\|x_u-x_v\|
.\] 
\end{document}

