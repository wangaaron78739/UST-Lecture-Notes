\documentclass[../main/main.tex]{subfiles}


\begin{document}

\section{May  7th, 2019}
\subsection{Well-Characterized Problems}
\begin{definition}
	\vocab{NP}: Class of decision problems for which there is a ``yes-certificate" (i.e. there for yes-instances, there is a "short" proof that the answer is yes)
\end{definition}
Note that this is not symmetric, as such, we have the CO-NP class
\begin{definition}
	\vocab{CO-NP}: Class of decision problems for which there is a ``no-certificate"
\end{definition}
We know that: \[
\text{Hamiltonian Cycle Problem}\in \text{NP}
.\] However, we still don't know if the Hamiltonian Cycle Problem is in CO-NP. The belief among experts is that: \[
\text{Hamiltonian Cycle Problem} \in \text{CO-NP}
.\] 
Note that $\text{P}\subset \text{CO-NP}$, since we can just solve the problem for a no-certificate.
\begin{figure}[htpb]
	\centering
	\includegraphics[width=0.6\textwidth]{5-7-zoo}
	\caption{$\text{NP}\cap \text{CO-NP}=\text{P}$?}
	\label{fig:5-7-zoo}
\end{figure}
\begin{definition}
	
	Problem that have both YES and NO certificates, i.e. problem in $\text{NP}\cap \text{CO-NP}$ are said to be \vocab{well characterized} or said to have a \vocab{good characterization}
\end{definition}
Let us consider the decision version of LP: Given a feasible and bounded maximization LP, is there a feasible solution whose value is at least $k$?\\

We already know that there is a YES-certificate, as we can just check the feasible solution whose value is at least $k$, i.e.  $\text{LP}\in \text{NP}$. 
\begin{remark}
	Note that we did not have to know that there is a poly-time solution for LP, as we are only looking at verifying the YES-certificates
\end{remark}
To see whether the LP problem lies in CO-NP, we can consider the dual of the LP. A NO-certificate would be any feasible solution of the dual whose value is less than $k$, meaning that $\text{LP}\in \text{CO-NP}$.\\

This means that LP is a well-characterized problem. Once again, note that we did not have to use the fact that there is a poly-time algorithm for LP. Historically, when a problem is found to be well-characterized, it is likely that it belongs to P.\\

Decision Problem for Max Flow: Given a flow network, is there a feasible flow of value greater than or equal to $k$? 
\begin{itemize}
	\item Feasible flow is a yes-certificate, so this problem belongs to NP. 
	\item The no-certificate is cut, as if there does not exist a flow greater than or equal to $k$, then there is a cut of value less than $k$. 
	\item Note that once again we have shown that this problem is well-characterized, without using the knowledge of having a poly-time solution.
\end{itemize}

Decision Problem for Bipartite Perfect Matching: Given a bipartite graph $G$ with bipartition  $(A,B)$ such that $|A|=|B|$, does $G $ have a perfect matching?
\begin{itemize}
	\item Yes-certificate is just the perfect matching which can be easily checked
	\item No-certificate: we can use Hall's Theorem, i.e. show that: \[
			\exists X\subseteq A, \text{ s.t. }|N(X)|<|X|
	.\] We could also use Konig's Theorem: \[
	\exists \text{ vertex cover of size }<n
	.\] 
\item As such this problem is well-characterized.
\end{itemize}
Problems that are well-characterized typically have an associated min-max relationship. These relationships are: 
\begin{itemize}
	\item beautiful and powerful cominatorial results
	\item lead to poly-time algorithms that are designed around it
	\item special cases of LP-duality.
\end{itemize} 
We will now look at a well-characterized problem for which we do not know if there is a poly-time algorithm yet - factorization.
\subsection{Well-characterization of Factorization}
The decision problem for factorization is: \[\text{Given two integers $x$ and $y$, does $x$ have a factor less than $y$ and greater than $1$.}\]
To find its first factor, find less than $x$ and greater than $1$. If no, then it is prime, if yes, then we can do binary search, but with $y=\frac{x}{2}$. After we find a factor, we can divide it and find the next factor.\\

Note that it takes $\log x$ calls to find a factor and since 2 is the smallest factor, there can be at most $\log x$ factors ( $2\times 2\times \ldots\times 2$). As such this will take $\log^2 x$ calls. Since the input is of the order $\log x$, we would be able to check the yes-certificate.\\

The no-certificate is the prime factorization of $x$.
 \begin{enumerate}
	\item Check each prime factor and see if it is $\ge y$.
	\item We must check if each factor is indeed prime. We can do this with a black box since there is a poly-time algorithm for checking whether a number is prime (Manidra Agarwal).
	\item Multiply them together to see if their product is $x$.
	\item After doing this, you should be convinced that the answer is no.
\end{enumerate}
\subsection{MAX-SAT Problem}
SAT (Satisfiability) was the first problem to be shown was NP-Complete  (Proved by Stephen Cook in the early 1970's)\\
\textbf{Input:}
\begin{itemize}
	\item 
$n$ boolean variables (each can be set True or False)
\item $m$ clauses: $C_1,C_2,\ldots,C_m$ (each is a disjunction of literals)
\end{itemize}
\textbf{Goal:} Find an assignment of true/false to the variables that maximizes the number of satisfied clauses.
\begin{remark}
The original satisfiability problem: Given any boolean formula include ``and'', ``or'', ``not'', can the formula be ``satisfied''?
\end{remark}
\begin{example}
	Let the boolean formula be:\[
		(x_1 \vee x_2 \vee \overline{x_3})\wedge (\overline{x_2}\vee x_4)
	\] Can we assign $x_i$ true or false to have the formula be true? 
\end{example}
It turns out there this is equivalent (poly-time transformation) from this problem to MAX-SAT.
\begin{definition}
	$x_1$ is a \vocab{variable}, and $\overline{x_1}$ is the \vocab{negation of a variable}. Collectively, they are \vocab{literals}
\end{definition}
\begin{definition}
	$\vee$: or, \vocab{disjunction}\\
	$\wedge$: and, \vocab{conjunction}
\end{definition}
\begin{definition}
	A \vocab{clause} is one or more literals combined with disjunctions.
\end{definition}
\begin{definition}
	A \vocab{formula} is one or mor clauses combined with conjunctions
\end{definition}
We can rewrite SAT as: Given a set of clauses, is there an assignment of true/false to variable, such that all the clauses are satisfied.
\begin{remark}
	This is clearly in the class NP, as if the answer is yes, we can just give the solution.
\end{remark}

MAX-SAT is the maximization version of this problem (note that this is also NP-Hard, as we can easily use this to solve SAT).\\

\subsection{Randomized Algorithm 1}
\subsubsection*{Algorithm}
Set each variable $x_i$ to true independently with probability $\frac{1}{2}$.
\subsubsection*{Analysis} 
\begin{itemize}
	\item Let $C_j=x_1\vee x_2\vee \ldots\vee x_k$. 
	\item The probability that the clause is satisfied is $1-\frac{1}{2^k}\ge \frac{1}{2}$.
	\item The expected number of clauses satisfied is: \[
	\ge \frac{1}{2}m\ge \frac{1}{2}OPT
	.\] 
\end{itemize}
\begin{remark}
	Consider the variable of the problem in which each clause has exactly $3$ literals (MAX-3SAT). For this, we have: \[
	\text{EXP \# of clauses satisfied} \ge \frac{7}{8}m
	,\] meaning that the approximation factor would be $\frac{7}{8}$.
\end{remark}
\begin{theorem}
	If there is a $\left( \frac{7}{8}+\epsilon \right) $- approximation algorithm for MAX-3SAT for any constant $\epsilon>0$, then P=NP.
\end{theorem}
This means that our simple algorithm is the best possible for MAX-3SAT.
\end{document}

