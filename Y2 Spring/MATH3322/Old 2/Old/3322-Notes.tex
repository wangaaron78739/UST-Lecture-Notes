\documentclass[12pt]{article}
\usepackage{tikz}
\usepackage{pgfplots}
\usepackage[utf8]{inputenc}
\usepackage[english]{babel}
\usepackage{xcolor}
\usepackage{textcomp}
\usepackage[hyphens,spaces,obeyspaces]{url}
\usepackage{listings}
\lstset{language=C++,
		basicstyle=\ttfamily,
		keywordstyle=\color{blue}\ttfamily,
		stringstyle=\color{red}\ttfamily,
		commentstyle=\color{green}\ttfamily,
		morecomment=[l][\color{magenta}]{\#}
}

\usepackage{bm}
\usepackage{amsmath,amsthm,amssymb,amsfonts}
\usepackage{mathtools}

\usepackage{booktabs}
\usepackage{array}
\usepackage{fancyhdr}
\usepackage[a4paper, margin=1in]{geometry}
\usepackage{multicol}
\usepackage{enumerate}
\usepackage{enumitem}
\setlist{nolistsep}
\usepackage{graphicx}
\usepackage{gensymb}
\usepackage{subcaption}
\usepackage{algorithm}
\usepackage{algpseudocode}
% \usepackage[noend]{algpseudocode}
\graphicspath{ {./images/} }
\usepackage[super]{nth}
\theoremstyle{definition}
\newtheorem{theorem}{Theorem}
\newtheorem{corollary}{Corollary}[theorem]
\newtheorem{definition}{Definition}
\newtheorem{lemma}{Lemma}

\newtheorem*{remark}{Remark}

\setcounter{secnumdepth}{0}

\newcommand{\N}{\mathbb{N}}
\newcommand{\Z}{\mathbb{Z}}
\newcommand{\R}{\mathbb{R}}
\newcommand{\C}{\mathbb{C}}
\newcommand{\x}{\bm{x}}
\newcommand{\y}{\bm{y}}
\newcommand{\mat}[1]{\mathbf{#1}}
\newcommand{\norm}[1]{\left\lVert#1\right\rVert}
\newcommand{\PreserveBackslash}[1]{\let\temp=\\#1\let\\=\temp}
\newcolumntype{C}[1]{>{\PreserveBackslash\centering}p{#1}}
\newcolumntype{R}[1]{>{\PreserveBackslash\raggedleft}p{#1}}
\newcolumntype{L}[1]{>{\PreserveBackslash\raggedright}p{#1}}

\pagestyle{fancy}
% \fancyhf{}
\chead{Aaron Wang}
\rhead{\nouppercase\leftmark}
\lhead{MATH3322 - Notes}
\cfoot{\thepage}
\title{
{\LARGE MATH3322 - Matrix Computation} \\
\textbf{\large Instructor: Jianfeng Cai} 
}

\date{}

\begin{document}
\maketitle\thispagestyle{fancy}
\tableofcontents
\newpage

\section{Eigenvalue Decomposition}

		\definition
	 Let $A \in \R^{n\times n}$ be a square matrix. A non-zero vector $x$ is an eigenvector of $A$ with  $\lambda\in \C$ being the corresponding eigevalue if: \[
		Ax=\lambda x		.\]  	

\begin{itemize}
		\itemsep0.3cm
				\item Even if $A$ is a real matrix, its eigenvalue and eigenvectors can be complex
				\item The set of eigenvalues of $A$ is called the spectrum of $A$. The spectral radius $\rho\left( A \right) $ is the maximum value $\left| \lambda \right| $ over all eigenvalues of $A$.
				\item If $\left(\lambda,x \right) $ is an eigenpair of $A$, then:
						\begin{align*}
								\left( \lambda^2,x \right) &\textrm{ is a eigenpair of }A^2\\
								\left( \lambda-\sigma,x \right) &\textrm{ is a eigenpair of }A-\sigma I\\
								\left( \frac{1}{\lambda-\sigma},x \right) &\textrm{ is a eigenpair of }\left( A-\sigma I \right) ^{-1}
						.\end{align*}
\begin{proof}
							Since $\left( \lambda,x \right) $ is an eigenpair of $A$, $Ax=\lambda x$ Multiplying both sides by $A$ from the left:\[
				A\cdot A=\lambda A x \implies A^2x=\lambda Ax=\lambda\cdot\lambda x=\lambda^2 x
				.\] 
				\[
						Ax-\sigma x=\lambda x-\sigma x \implies \left( A-\sigma I \right) x=\left( \lambda-\sigma\right)x\]\[\implies x=\left( \lambda-\sigma \right) \left( A-\sigma I \right) ^{-1}x\implies\left( A-\sigma I \right) ^{-1}x
				.\]		
\end{proof}
\end{itemize}
\begin{definition}
		Two matricies $A$ and $B$ are similar with each other if there exists a nonsingular matrix $T$ such that \[
		B=TAT^{-1}
		.\]  
\end{definition}
\begin{theorem}
		If $A$ and $B$ are similar, then $A$ and $B$ have the same eigenvalues.
\end{theorem}
\begin{proof}
		Since $A,B$ are similar, $B=TAT^{-1}$, which implies $A=T^{-1}BT$. If $\left( \lambda, x \right) $ is an eigenpair of $A$, then $Ax=\lambda x$, so that \[
				T^{-1}BTx=\lambda x\implies B\left( Tx \right) =\lambda\left( Tx \right) 
		.\] Thus, $\left( \lambda, Tx \right) $ is an eigenpair of $B$. i.e. any eigenvalue of $A$ is an eigenvalue of $B$. The reverse is similar.
\end{proof}
\begin{definition}
		An eigenvalue decomposition of a square matrix $A\in \R^{n\times n}$ is a factorization \[
		A=X\Lambda X^{-1}
		,\] where $X\in\C^{n\times n}$ is non-singular and $\Lambda\in\C^{n\times n}$ is diagonal.
\end{definition}
\begin{itemize}
		\item If $A\in \R^{n\times n}$ admits an eigenvalue decomposition, then \[
		AX=X\Lambda
.\] If we rewrite $X=[x_1\space x_2\space \ldots\space x_n]$
\end{itemize}
\newpage
\section{ Decomposition}

\end{document}

