\documentclass[../main/main.tex]{subfiles}


\begin{document}

\section{November  3th, 2020}
\subsection{Error Analysis of Finite Element Methods}
For finite different methods, we can get the error analysis using the Taylor expansion. However, for FEM, it is a bit more complicated. Recall that the truncation error is when we truncate the infinitely dimensional solution space $H^1_0(\Omega)$ into a finite-dimensional subspace  $S_N \subset H^1_0(\Omega)$. Recall that for the problem: \[
\begin{cases}
    -\Delta u = f \\
u\big\rvert_{\partial \Omega} = 0
\end{cases}
.\] we have an equivalent weak formulation of finding $u \in  H^1_0(\Omega)$ such that: \[
\int_\Omega \nabla u\nabla v~dx = \int_\Omega fv, \quad  \forall  v \in  H^1_0(\Omega)
.\] Let us define a \vocab{bi-linear form}: \[
a(u,v) = \int_\Omega \nabla u \nabla  v~dx
.\] 
\begin{remark}
    Bi linear form means that $a(u,v)$ is linear in both  $u$ and $v$.
\end{remark}
Let us also define a norm: \[
    \|v\|_E = \sqrt{a(v,v)} = \left( \int_\Omega (\nabla v)^2~dx \right)  ^\frac{1}{2}
.\] 
Let us consider a finite dimensional subspace $S_N\subset H^1_0(\Omega)$, meaning that the finite element solution $u_N$ is such function that satisfies:  \[
    a(u_N,v) = (f,v), \quad\forall  v \in S_N
    .\] where $(u,v) = \int_\Omega uv$ is the inner product. For the exact solution $u$, we would have: \[
a(u,v) = (f,v) \quad \forall v \in  S_N
.\] \[
\implies a(u-u_N,v) = 0 \quad\forall  v\in S_N\subset H^1_0(\Omega)
.\] Let us consider the error, i.e. the difference between $u$ and $u_N$:
\begin{align*} 
    \|u-u_N\|^2_E &= a(u-u_N,u-u_N) \\
                  &= a(u-u_N, u-v+v-u_N), \quad \forall  v \in  S_N\\
                  &= a(u-u_N,u-v) + a(u-u_N,v-u_N) \\
                  &= a(u-u_N,u-v) \quad\quad (\text{since $v-u_N\in S_N$})\\
                  &\le \|u-u_N\|_E \|u-v\|_E, \quad \forall  v \in  S_N \quad\text{(Cauchy Schwartz inequality)}
.\end{align*}
As such, we have: \[
    \|u-u_{N}\|_E \le  \|u-v\|_E \quad\forall v \in S_N
.\] This means that out of all solutions in $S_N$, $u_N$ is the best approximation to the exact solution. \\

To estimate the error, we will use a duality argument. Let $w$ be a solution of the dual problem:  \[
\begin{cases}
    -\Delta w = u-u_N\\
    w(0) = w(1) = 0
\end{cases}
.\] Let us consider the $L^2$ norm: \[
\|v`\|_{L^2} = \left( \int_\Omega |v|^2~dx \right) ^\frac{1}{2}
.\]we have:
\begin{align*} 
    \|u-u_N\|^2_{L^2} &= \int |u-u_N|^2 ~dx \\
                      &= (u-u_N,u-u_N)\\
                      &= (u-u_N, -w'') \\
                      &= ((u-u_N)',w') \\
                      &= a(u-u_N,w) \\
                      &= a(u-u_N,w-v+v),\quad\forall v \in  S_N \\
                      &= (u-u_N,w-v) + a(u-u_N,v) \\
                      &\le \|u-u_N\|_E \|w-v\|_E 
.\end{align*}
As such, we have: \[
\|u-u_N\|_{2} \le  \frac{\|u-u_N\|_E \|w-v\|_E}{\|u-u_N\|_2} = \frac{\|u-u_N\|_E \|w-v\|_E}{\|w\|_2}, \quad \forall  v \in S_N 
.\] If we have the approximation assumption: \[
\int_{w \in  S_N} \|w-v\|_E \le  \epsilon \|w''\|_2 \implies \|u-u_N\|_2 \le \epsilon \|u-u_N\|_E 
.\] 
\begin{remark}
    This approximation assumption is that any second order differentiable function can be approximated by a function in $S_N$. 
\end{remark}
Applying the approximation assumption again, we have: \[
\|u-u_N\|_E \le  \epsilon \|u-v\|_E \le  \epsilon \|u''\|_2
.\] \[
\implies \|u-u_N\|_2 \le  \epsilon^2  \|u''\|_2 = \epsilon^2\|f\|_2
.\] As such, it is second order accurate in $ L^2$ norm.\\

    Consider $0=x_0<x_1<x_2<\ldots<x_N=1$ a partition of $[0,1]$. Let $S_N$ be the continuous piecewise linear function space with hat basis functions  $\phi_1,\ldots,\phi_{N-  1}$. Take any function $u\in H^1_0(\Omega)$. Let $u_I(x) = \sum_{i=1}^{N-1} u(x_i)\phi_i(x)$.
    \begin{theorem}
        \[
            h=\max_i (x_{i+1}-x_i) \implies \|u-u_I\|_E \le  c h \|u''\|_2
        .\] 
    \end{theorem}
    \begin{proof}
       \[
           \|u-u_I\|_E^2 = \int^1_0 (u'-u_I')^2 ~dx = \sum_{j=1}^{N} \int ^{x_j}_{x_{j-1}} (u'-u_I')^2 ~dx
       .\]  
       Let $e(x) = u(x) - u_I(0)$, with $e(x_j) = 0$. For  $[x_{j-1},x_j]$: \[
           \exists \xi \in (x_{j-1},x_j),\text{ such that } e'(\xi) = 0, e'(y) = \int^y_{\xi}e''(x)~dx 
       .\] from the fundamental theorem of calculus.
       As such, we have: \[
           |e'(y)| \le  \int^y_\xi |e''(x)|~dx \le  \left( \int_\xi^y ~dx \right)^\frac{1}{2} \left( \int^y_\xi |e''(x)|^2~dx \right) ^\frac{1}{2}
       .\] \[
       = (y-\xi)^{\frac{1}{2}} \left( \int^y_\xi|e''(x)|^2~dx \right) ^{\frac{1}{2}} \le  (y-\xi)^{\frac{1}{2}}\left( \int ^{x_j}_{x_{j-1}}|e''(x)|^2~dx\right)^\frac{1}{2}
       .\] Thus, we have: 
\begin{align*} 
    \int ^{x_j}_{x_{j-1}}|e'(y)|~dy &\le  \int ^{x_j}_{x_{j-1}} (y-\xi))\left( \int ^{x_j}_{x_{j-1}}|e''(x)|^2~dx \right) ~dy\\
                                    &= \frac{1}{2}(y-\xi)^2\bigg\rvert ^{x_j}_{x_{j-1}}\int ^{x_j}_{x_{j-1}} |e''(x)|^2~dx \\
                                    &\le c(x_j-x_{j-1})^2\int ^{x_j}_{x_{j-1}} |e''(y)|^2~dx\quad \text{ for some constant $c$}\\
                                    &\le c h^2\int ^{x_j}_{x_{j-1}}(u'')^2~dx \quad\text{since $e'' = u''$}
.\end{align*}
Thus, we have: \[
    \left( \int^1_0(u'-u'_I)^2~dy \right)^{\frac{1}{2}}
    \le c h \left( \int^1_0 (u'')^2 ~dx \right)^{\frac{1}{2}}
.\] \[
\implies \|u-u_I\|_E \le  ch \|u''\|_2
.\] If we choose $v=w_I \in  S_N$ to be the interpolation function of $w$, we would have: \[
\| w-v\|_E = \|w- w_I\|_E \le  ch \| w''\|_2
.\]
    \end{proof} As such, from before, we would have: \[ \|u-u_N\|_2 \le  ch \|u-u_N\|_E \le ch^2\|u''\|_2 = ch^2 \|f\|_2
.\] 
As such, this finite element scheme is second order accurate.
\begin{remark}
    We can increase this accuracy by choosing different basis functions, e.g. piecewise quadratic functions.
\end{remark}
\end{document}

