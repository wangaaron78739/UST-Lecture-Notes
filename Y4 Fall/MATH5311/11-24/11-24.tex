\documentclass[../main/main.tex]{subfiles}


\begin{document}

\section{November 24th, 2020}
\subsection{Spectral Methods}
Spectral methods are based on Fourier series and have high accuracy. If we are the domain $[0,2\pi]$, we can define the periodic functions:  \[
    \phi_k(x) = e^{ikx} = \cos kx + i \sin kx
.\] For any function $u(x)\in L^2[0,2\pi$, we can express it in terms of its Fourier series: \[
u(x) = \sum_{k=-\infty}^{+\infty} u_k e^{ikx}
.\] Where: \[
u_k = \frac{1}{2\pi}\int_0^{2\pi} u(x) e ^{-ikx}dx
.\] $u_k$ are called the \vocab{fourier coefficients}.

\begin{remark}
    $L^2$ function means it is square integrable.
\end{remark}
    $\phi_k(x)$ are orthogonal as:  \[
        \int_0^{2\pi}\phi_k \overline{\phi_\ell} dx = \int_0^{2\pi} e^{ikx}e^{-i\ell x}dx = \int ^{2\pi}_0 e^{i(k-\ell) x}dx = \begin{cases}
            2\pi & k = \ell \\
            0 & k \neq \ell
        \end{cases}
    .\] 
    $\phi_k(x)$ are complete, meaning for any  $u\in L^2[0,2\pi]$ it can be approximated by: \[
        u(x)\sim \sum_{k=-N}^{N} u_k \phi_k(x
    .\] in the sense that: \[
    \lim_{N\to \infty} \|u(x) - \sum_{k=-N}^{N} u_k \phi_k(x)\| = 0
    .\] 
    As such $\{\phi_k\} $ forms a basis of $L^2[0,2\pi]$.

    \begin{definition}[Spectral Projection]
        Given a function $u(x) \in  L^2[0,2\pi]$, we define: \[
            P_Nu(x) = \sum_{k=-N}^N u_k e^{ikx}
        \] to be the \vocab{spectral projection} of $u$.  
    \end{definition}
    \begin{remark}
        In other words, the spectral projection is the finite truncation of the Fourier series.
    \end{remark}
    This spectral projection has a few properties:
    \begin{itemize}
        \item $P_N$ is a projection, i.e.  $P_N^2 = P_N$.
            \begin{proof}
                Let $\tilde{u}(x) = P_Nu(x)$
                We have: \[
                P_N^2 u = P_N \tilde{u}
                .\] with 
                .
                \begin{align*}
                    \tilde{u_\ell}(x) &= \frac{1}{2\pi} \int ^{2\pi}_0 \tilde{u}(x) e^{-i\ell x}dx \\
                                      &= \frac{1}{2\pi} \int ^{2\pi}_0 \left( \sum_{k = -N}^{N}u_k e^{ikx} \right)  e^{-i\ell x}dx \\
                                      &= \frac{1}{2\pi}\sum_{k=-N}^Nu_k \int ^{2\pi}_0 e^{ikx}  e^{-i\ell x}dx \\
                                      &= u_\ell
                .\end{align*}
                by using the orthogonality of $\phi_k$.
            \end{proof}
        \item $\|P_N u(x) \|^2_{L^2} = 2\pi \sum |u_k|^2$.
            \begin{proof}
                We have: 
                \begin{align*}
                    \|P_N u(x)\|^2_{L^2} &= \int ^{2\pi}_0 |P_Nu|^2 dx \\
                                     &= \int ^{2\pi}_0 \left( \sum_{k=-N}^{N} u_k e^{ikx}\right)\left( \sum_{\ell=-N}^{N} \overline{u_\ell} e^{i\ell x}\right) dx \\
                                     &= \int ^{2\pi}_0 \sum_{k=-N}^{N}\sum_{\ell=-N}^{N} u_k \overline{u_\ell} e^{i(k-\ell)x}dx \\
                                     &= \sum_k\sum_\ell u_k \overline{u_\ell} \int ^{2\pi}_0 e^{i(k-\ell) x}dx \\
                                     &= 2\pi \sum_k |u_k|^2 
                .\end{align*} again by using the orthogonality of $\phi_k$.  
            \end{proof}
    \end{itemize}
    \subsection{Decay Rate of Spectral Methods}
    Let us consider the decay rates of the coefficients: \[
        u_k = \frac{1}{2\pi}\int ^{2\pi}_0 u(x) e^{-ikx}dx
    .\] If $u$ is differentiable, then do integration by parts giving us: \[
u_k = -\frac{1}{2\pi} \frac{1}{ik}u(x) e^{-ikx}\bigg\rvert ^{2\pi}_0 + \frac{1}{ik} \frac{1}{2\pi}\int ^{2\pi}_0u'(x) e^{-ikx}dx
    .\]  Because $u$ is periodic, the first term is 0, meaning that: \[
    u_k = \frac{1}{ik} \frac{1}{2\pi}\int ^{2\pi}_0 u'(x) e^{-ikx}dx 
    .\] Note that this is $\frac{1}{ik}(u')_k$. This can be continued, giving us: \[
    u_k = \frac{1}{(ik)^r} \frac{1}{2\pi}\int ^{2\pi}_0 u^{(r)}e^{-ikx}dx
    .\]  Taking the absolute value, we have: \[
    |u_k| \le  \frac{1}{|k|^r} \frac{1}{2\pi} \int ^{2\pi}_0 |u^{(r)}| dx \le  \frac{C}{|k|^r} \| u^{(r)}\|_{L^2}
    .\] 
    Note that this goes to zero, as $k$ goes to infinity. Note that since $r$ is arbitrary, this goes to zero faster than any power, as we can just increase  $r$. As such, it goes to zero exponentially fast. Of course, this requires that the function is differentiable to any order. \\

    We will now prove this exponential convergence of Fourier series for smooth functions.
    \begin{proof}
        Let $u(x)$ be smooth, i.e. differentiable to any order, then: \[
            u(x) = \sum_{k=-\infty}^{+\infty}u_k e^{ikx}, \quad u_k = \frac{1}{2\pi} \int ^{2\pi}_0 u(x) e^{-ikx}dx
        .\] Note that $u(x)$ is absolutely convergent, as  $u_k$ decays faster than any power. As such, we have:  \[
        |u(x) | \le  \sum_k |u_k| 
        .\] If $u_k \le  \frac{1}{|k|^2}$, then the right hand side is convergent. Now let's consider: \[
        \|u(x) - \sum_{k=-N}^N u_k e^{ikx}\|_{\infty}
        .\] which converges to $0$ as $N\to  +\infty$. We have: 
        \begin{align*} 
        \|u(x) - \sum_{k=-N}^N u_k e^{ikx}\|_\infty &= \|\sum_{|k| > N}u_k e^{ikx}\|_\infty \\
        &\le \sum_{|K| > N} |u_k| |e^{ikx}| \\
        &= \sum_{|k| > N} |u_k| \\
        &\le  \sum_{|k| > N} \frac{1}{|k|^r} \|u^{(r)}\| \\
        &\le  C \sum_{|k| > N} \frac{1}{|k|^r} \\
        &= \frac{C}{(N+1)^r} \left( \sum_{j = 1} \left( \frac{N+1}{N+j} \right)^ r \right)  \\
        &\le  \frac{C}{(N+1)^r} \cdot  C_1
        .\end{align*}
        This means that: \[
            \|u(x) - P_Nu(x)\| \le  \frac{M}{(N+1)^r} \to  0
        .\] for some constant $M$, which converges faster than any power. As such, it has exponential convergence.
\end{proof}
         \begin{remark}
            How fast the function decays in the frequency space depends on how smooth it is in the physical space.
        \end{remark}
        \begin{definition}[Total Variation]
            Let us define $u(x)$ on  $[0,2\pi]$ with a mesh $0=x_0\le  x_1 \le  x_2 \le  \ldots \le x_n \le  x_{n+1} = 2\pi$. We define the \vocab{total variation} as: \[
                \gamma(u) = \sup_n\sup_{\{x_i\} }\sum_i |u(x_i) - u(x_{i-1}) |
            .\] 
            
        \end{definition}
        \begin{example}
            If $u(x)$ is a monotone increasing function on  $[0,2\pi]$, then we have $\gamma(u) = u(2\pi) - u(0)$.
        \end{example}
        \begin{remark}
            If $u(x)$ is not monotone $[0,2\pi]$, then if we can split the region into monotone sections, we can calculate the total variation as the absolute value of all the difference in the monotone sections.
        \end{remark}
        \begin{example}
            Consider $u(x) = \sin x$. We can split it into 3 sections, with the total variation being $\gamma(u) = 4$.
        \end{example}
        \begin{example}
            If $u(x) = \sin \frac{1}{x}$, then $u(x)$ has no bounded variation.
        \end{example}
            Bounded variation is a very important property to determine the convergence of its Fourier series. It has the following properites: 
            \begin{itemize}
                \item If $u$ is continuous, periodic, and has bounded variation, then we have: $P_Nu(x) \to  u(x) $ uniformly.
                \item If $u(x)$ has bounded variation, then  $P_Nu(x) \to  \frac{1}{2}(u^+(x) + u^-(x))$, which is the average of its left and right limits (note that it does not need to be continuous), this is pointwise convergence.
                \item If $u \in  L^2[0,2\pi]$, then: \[
                        \|P_Nu(x) - u(x)\|_{L^2} = \int ^{2\pi}_0 |P_N u(x) - u(x)|^2dx \to  0 
                .\] This means it converges in the average sense.
            \item If $u(x)$ is continuous and periodic, but not of bounded variation, then $P_Nu(x)$ does not  necessarily converge for all $x$, e.g.  $u(x) = x\sin \frac{1}{x}, x \in [-\frac{1}{\pi} , \frac{1}{\pi}]$. At $x=0$,  $P_Nu(x)$ is not convergent.
            \end{itemize} 
            \begin{remark}
                Uniform convergence means that: \[
                    \lim\limits_{N \to \infty} \max_{x \in [0,2\pi]} |u(x) - P_N u(x) | = 0
                .\] 
                Pointwise convergences means that for any $x \in [0,2\pi]$: \[
                    \lim\limits_{N \to \infty} |u(x) - P_Nu(x) | = 0
                .\] 
            \end{remark}
\end{document}

