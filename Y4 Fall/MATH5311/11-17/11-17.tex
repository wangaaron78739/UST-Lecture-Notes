\documentclass[../main/main.tex]{subfiles}


\begin{document}

\section{November 17th, 2020}
\subsection{Methods to solve PDEs}
Last time, we tried to calculate the explicit scheme for some iterative schemes. Let us consider the model: \[
\begin{cases}
    -u''(x) + \sigma u(x) = f(x) & 0 < x < 1, \sigma > 0\\
    u(0) = u(1) = 0
\end{cases}
.\] 
If we use a uniform grid with: \[
h = \frac{1}{N}, x_i = ih
.\] We can use the Taylor expansion to approximate the second derivative using central different with second order accuracy. With this, we can approximate the boundary value problem with a finite difference scheme, giving us a symmetric positive definite matrix equation that we solve iteratively. 

There are a few other methods to solve this: 
\begin{itemize}
    \item Direct Methods
        \begin{itemize}
            \item Gaussian elimination
            \item Factorization
        \end{itemize}
    \item Iterative Methods
        \begin{itemize}
            \item Jacobi
            \item Gauss-Seidel
            \item SOR
            \item Conjugate Gradient
        \end{itemize}
\end{itemize}

If we have a 2D problem, we would have a similar situation with a linear block tridiagonal system.  

Coming back to the iterative method, consider $Au=f$ where $A$ is NxN and let $v$ be an approximation of $u$. There are two important measures: 
\begin{itemize}
    \item The error: $e=u-v$ with norms: \[
    \|e\|_\infty = \max |e_i| \quad \|e\|_2 = \sqrt{\sum_{i=1}^{N} e_i^2} 
    .\] 
\item The residual: $r=f-Av$ with norms: \[
\|r\|_\infty \quad \|r\|_2
.\] 
\end{itemize}
We can rewrite $Au=f$ as: \[
    A(v+e) = f
.\] which means that \[
Ae = f - Av = r
.\] which is called the \vocab{residual equation}. This means that if we know the error we can correct it with $u=v+e$. This is called \vocab{residual correction}.

\subsection{Relaxation Schemes}
The Jacobi and Gauss-Seidel iterations are both relaxation schemes. 

Let us consider the weighted Jacobi relaxation: \[
    v^{n+1}_i = (1-\omega) v^{n}_i + \frac{\omega}{2}(v^{n}_{i-1} + v^{n}_{i+1} + h^2 f_i), \quad0 < \omega < 2
.\] 
\begin{remark}
    When $\omega=1$, this is the Jacobi iteration.
\end{remark}

There a few other version as well


\end{document}

