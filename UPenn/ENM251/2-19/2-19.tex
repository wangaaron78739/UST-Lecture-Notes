\documentclass[../main/main.tex]{subfiles}

%\usepackage{circuitikz}

\begin{document}

\section{February 19th, 2020}
\subsection{Unit Step Function Continued}
As a reminder, the unit step function is defined as: \[
	u(t-a) = \begin{cases}
		0,\quad t<a \\
		1,\quad t>a
	\end{cases}
.\] Given a piecewise function, we can write it as a linear combination of step functions. 
\begin{example} \label{2-19-1}
	Consider: \[
		f(t) = \begin{cases}
			7,\quad 0 < t < 2 \\
			6t,\quad 2 < t < 3\\
			t^2, \quad 3 < t < 7 \\
			0, \quad 7 < t
		\end{cases} 
	.\] We can rewrite this as: \[
	f(t) = 7u(t) + (6t-7)u(t-2) + (t^2-6t) u(t-3) + (0-t^2)u(t-7)
	.\] 
\end{example}
With this, we can take the Laplace transform of the function, but first, we need to consider the Laplace transform of $\La \{f(t) u(t-a)\} $. Looking at the definition, we have: \[
	\La \{f(t) u(t-a)\} = \int^\infty_0 f(t) u(t-a) e^{-st}~dt
.\] Assuming $a> 0$, we have: \[
= \int^a_0 f(t) \underbrace{u(t-a)}_0 e^{-st}~dt + \int^\infty_a f(t) \underbrace{u(t-a)}_1 e^{-st}~dt = \int^\infty_a f(t) e^{-st}~dt
.\] If we set $z=t-a \implies dz = dt$, \[
= \int_0^\infty f(z+a) e^{-s(z+a)}~dz = e^{-as}\La \{f(t+a)\} 
.\] 
\begin{theorem}[Shifting Theorem]\index{shifting theorem}
	As shown above: \[
		\La \{f(t) u(t-a)\} =e^{-as}\La \{f(t+a)\} 
	.\] 
\end{theorem}
\begin{example}
	Considering $f(t)$ from Example \ref{2-19-1}, we have: \[
		\La \{f(t)\} = \int^2_0 7 e^{-st}~dt + \int^3_2 6t e^{-st}~dt + \int^7_3 t^2e^{-st}~dt + \int^\infty_7 0 e^{-st}~dt
	.\] However, we can calculate this another way. From the table, we have $\La \{t ^{m}\} = \frac{m!}{s^{m+1}}$, thus: \[
	\La \{f(t)\} = \La \{7u(t) + (6t-7)u(t-2) + (t^2-6t)u(t-3) + (0-t^2)u(t-7)\} 
	\] \[
	=\La \{7u(t)\} + \La \{(6t-7)u(t-2)\} +\La \{(t ^2-6t)u(t-3)\} +\La \{-t^2u(t-7)\} 
	\] \[
	= e^{-0s}\La \{ 7\} + e^{-2s}\La \{6(t+2)-7\} + e^{-3s}\La \{(t+3)^2-6(t+3)\} -e^{-7s}\La \{(t+7)^2\}  
	\] \[
	= \frac{7}{s}+e^{-2s}\La \{6t-5\} +e^{-3s}\La \{t^2-9\} -e^{-7s}\La \{t^2+14t+49\} 
	.\] Thus: \[
	\La \{f(t)\} =\frac{7}{s}+e^{-2s}\left( \frac{6}{s^2}+\frac{5}{s} \right) + e^{-3s}\left( \frac{2}{s^{3}}-\frac{9}{s} \right) -e^{-7s}\left( \frac{2}{s^{3}}+\frac{14}{s^2}+\frac{49}{s} \right) ,\quad s>0
	.\]  
\end{example}
\begin{remark}
	In the example earlier, we are using the Shifting Theorem and replacing $t$ with $t+a$ in each of the functions that we are multiplying by the unit step function at  $a$.
\end{remark}
\subsection{Examples of Where Unit Step Functions Occur}
\begin{example}
	Consider the equation: \[
		L \frac{dI}{dt}+RI= \epsilon_1u(t) + (\epsilon_2-\epsilon_1)u(t-t_1)
	.\] Taking the Laplace transform of both sides, we get: \[
	\La \left\{L\frac{dI}{dt}+RI\right\} = \La \{\epsilon_1u(t) + (\epsilon_2-\epsilon_1)u(t-t_1)\}  
	.\] \[
	\implies L \La\{I'(t)\} + R\La \{I(t)\} =e^{-0s}\La \{\epsilon_1\} + e^{-t_1s}\La \{\epsilon_2-\epsilon_1\}  
	\]\[
	\implies L\left( s\La \{I\} -I(0) \right) + R\La \{I\} = \frac{\epsilon_1}{s}+e^{-t_1s}\left( \frac{\epsilon_2-\epsilon_1}{a} \right) 
	.\]  \[
	\implies \La \{I\} = \frac{L I_0 + \frac{\epsilon_1}{s}+e^{-t_1s}\left( \frac{\epsilon_2-\epsilon_1}{s} \right) }{LS+R} 
	.\] 
\end{example}
There are many applications/cases where using a step function to describe a piecewise function might be useful. For example, if we have a spring with dampener with an external force $F(t)$, we might have $F(t)$ ramp up with $t$, and then stay constant after a certain amount of time.

Another example is consider a ball bouncing off the ground. The forces are: \[
	F(t) = \begin{cases}
		-mg,\quad 0<t<T_F\\
		N(t)-mg,\quad T_F < t < T_F+T_C \\ 
		-mg, T_F + T_C < t < T_F + T_C + T_R
	\end{cases} 
.\] Where $T_F$ is the time until hitting the ground, $T_C$ is the contact duration, and $T_R$ is the time to rebound back up, and $N(t)$ is the normal force. From this, we get figure out $N(t)$ and allow us to get the coefficient of restitution. 

\subsection{Impulse Function}
Consider a function: \[
	I_a(t) = \begin{cases}
		0,\quad t<-\frac{a}{2}\\
		\frac{1}{a},\quad -\frac{a}{2}<t<\frac{a}{2}\\
		0,\quad \frac{a}{2}<t\\
	\end{cases}
.\] This can be expressed in terms of unit step functions as: \[
I_a(t) = \frac{1}{a}u(t+\frac{a}{2}) - \frac{1}{a}u(t-\frac{a}{2})
.\] 
\begin{remark} 
Note that the area under the curve is $1$, as we choose the height to be inversely proportional to the width. This means that: \[
	\int^\infty_{-\infty}I_a(t)~dt = 1
.\]
\end{remark}

\begin{definition}[Impulse Function]
	An \vocab{impulse function} \index{impulse function} is: \[
		\lim\limits_{a \to 0} I_a(t) = I(t) = \begin{cases}
			0,\quad t\neq 0\\
			+\infty,\quad t=0
		\end{cases}
	\] with the property: \[
	\int^\infty_{-\infty}I(t) ~dt = 1
	.\] Or: \[
	\int_R I(t)~dt = \begin{cases}
		0,\quad 0\not\in R\\
		1,\quad 0 \in R
	\end{cases}
	.\] 

\end{definition}
\begin{remark}
	With Laplace transform, if $0$ is at the end of the domain, it is included, e.g.: \[
		\int^7_0 I(t)~dt = 1
	.\] 
\end{remark}
\begin{remark}
	Similar to the step function, we can shift the impulse function, i.e.: \[
		I(t-a) = \begin{cases}
			0,\quad t\neq a\\
			+\infty,\quad t=a
		\end{cases}
	,\] with: \[
	\int_R I(t-a)~dt = \begin{cases}
		0,\quad a\not\in R\\ 1,\quad a \in R
	\end{cases}
	.\]  
\end{remark}
\begin{example}[One Dimensional Crystal]
	In a one dimensional crystal, we have atoms aligned in a line, and say they are separated by $a$. If we have an electron travelling along, the force it might see can be expressed as: \[
		F(x) = a\sum_{k=-\infty}^{\infty} F_0 I(x-ka) = aF_0 \sum_{k=-\infty}^{\infty} I(x-ka) 
	.\] Thus one way to model the force experienced by an electron is to use a bunch of impulse functions. This is called the \vocab{comb function}.
\end{example}
\begin{remark}
	If we had a continuous function $f(t)$, we'd have: \[
		\int_R f(t) I(t-a)~dt = \begin{cases}
			0,\quad 0\not\in R \\ f(a),\quad a\in  R
		\end{cases}
	.\] 
\end{remark}
\begin{example}
	If we have: \[
		\int^7_{-1}\frac{t^2}{\sqrt{3t ^{3}+1} }e^{t}I(t-1)~dt = \frac{1^2}{\sqrt{3(1)^3+1} }e^{1} = \frac{1}{2}e
	.\] 
\end{example}
\begin{theorem}
	We have: \[
		\La \{f(t) I(t-a)\} ,a>0 = \int^\infty_0 f(t) I(t-a) e^{-st}~dt = f(a)e^{-as}
	.\] 
\end{theorem}
\begin{example}
	$ \La \{t ^{3}I(t-4)\} = 4^{3}e^{-4s}=64 e^{-4s}$ 
\end{example}
\begin{example}
	Consider $F(t) = aF \sum_{k=-\infty}^{\infty} I(t-ka)$. We have: \[
		\La \{F(t)\}  = Fa \sum_{k=-\infty}^{\infty} e^{-kas}
	.\] 
\end{example}

%\begin{figure}[h!]
	%\centering
	%\begin{tikzpicture}
	%\begin{axis}[width=6cm,height=3cm,domain=0:1,samples=501,ymin=0,ymax=1,xmin=-1,xmax=1,axis lines=center,
	%xtick=\empty,ytick=\empty,
	%xlabel={$t$},xlabel style={anchor=west},
	%ylabel={$f(t)$},ylabel style={anchor=south}]
	%\addplot[ultra thick,blue,mark=*,mark options={fill=white},samples at={-1.1,0,5}] {0};
	%\addplot[ultra thick,blue,mark=*,mark options={fill=white},samples at={0,0}] {1};
	%\end{axis}
%\end{tikzpicture}
	%\caption{ Example of an Impulse Function Funciton}
	%\label{fig:}
%\end{figure}	


%In this example, we have a circuit with equation: \[
	%L \frac{di}{dt}+RI = \epsilon_1 u(t) + (\epsilon-\epsilon_1)u(t-t_1)
%.\] Note that the RHS is: \[
%\text{RHS}= \begin{cases}
	%\epsilon_1,\quad 0<t < t_1\\
	%\epsilon_2,\quad t_1<t
%\end{cases}
%.\] 




%\begin{figure}[h!]
  %\begin{center}
    %\begin{circuitikz}
      %\draw (0,0)
      %to[C=$\epsilon_q$] (0,2) % The voltage source
      %to[short] (2,2)
      %to[R=$R_1$] (2,0) % The resistor
      %to[short] (0,0);
    %\end{circuitikz}
    %\caption{My first circuit.}
  %\end{center}
%\end{figure}



%\begin{figure}[h!]
	%\centering
	%\begin{tikzpicture}
    %\begin{axis}[width=6cm,height=3cm,domain=0:1,samples=501,ymin=0,ymax=1,xmin=-1,xmax=2,axis lines=center,
    %xtick=\empty,ytick=\empty,
    %xlabel={$t$},xlabel style={anchor=west},
    %ylabel={$f(t)$},ylabel style={anchor=south}]
	%\addplot[ultra thick,blue,mark=*,mark options={fill=white},samples at={-1.1,1}] {0};
%\addplot[ultra thick,blue,mark=*,samples at={1,2.1}] {0.5};
    %\end{axis}
%\end{tikzpicture}
	%\caption{ Example of a Unit Step Funciton}
	%\label{fig:}
%\end{figure}	
\end{document}

