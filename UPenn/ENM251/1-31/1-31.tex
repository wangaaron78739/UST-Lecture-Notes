\documentclass[../main/main.tex]{subfiles}


\definecolor{dkgreen}{rgb}{0,0.6,0}
\definecolor{gray}{rgb}{0.5,0.5,0.5}
\definecolor{mauve}{rgb}{0.58,0,0.82}

\lstset{frame=tb,
  language=MATLAB,
  aboveskip=3mm,
  belowskip=3mm,
  showstringspaces=false,
  columns=flexible,
  basicstyle={\small\ttfamily},
  numbers=none,
  numberstyle=\tiny\color{gray},
  keywordstyle=\color{blue},
  commentstyle=\color{dkgreen},
  stringstyle=\color{mauve},
  breaklines=true,
  breakatwhitespace=true,
  tabsize=3
}

\begin{document}

\section{January 31st, 2020}
\subsection{Problem 1}
Find period of motion for the equation: \[
	\dot{\theta}= \sqrt{\frac{g}{L}(3+2\cos\theta)}\quad 0\le \theta\le 2\pi
.\] Since the RHS only has $\theta$, this is separable, thus:  \[
\int	dt=\sqrt{\frac{L}{g}}\int \frac{d\theta}{\sqrt{3+2\cos(\theta)} }
\] Note that the RHS gives us an elliptical equation. Since we want the period, we have: \[
T = \sqrt{\frac{L}{g}} \int_0^{2\pi} \frac{d\theta}{3+2\cos\theta}+C
.\] We can consider $C$ to be the start time, and just set it to 0. This is as far as you can go analytically, so plug it into a calculator. 
\subsubsection{How to use in MATLAB}
\begin{lstlisting}
T = integral(@(theta)1./sqrCos(1,theta),2,2*pi)
tspan = [0 2.5];
y0 = 0;
data = ode45(@sqrCos,tspan,y0);

function res = sqrCos(t,theta)
	L = 2,4;
	g = 9,8;
	res = sqrt(g/L*(3+2*cos(theta)));
end(function)
\end{lstlisting}
\subsection{Problem 3}
Consider the equation  \[
v \frac{dv}{dx}+\frac{v^2}{x+\frac{m}{\rho}}=g
.\] With the initial condition: $v_0=v(x_0)=v(0)=0$. To solve for $v(x)$, note that this is a Bernoulli equation: \[
\frac{dv}{dx}+\frac{1}{x+\frac{m}{\rho}}v=g^{v-1}
.\] with: \[
p(x) = \frac{1}{x+\frac{m}{\rho}}\quad Q(x)=g \quad n=-1
.\] Plugging into the formula, we have: \[
V(x) = \left( \frac{1}{\mu(x)}\left( \int(1-n)\mu(x)Q(x)dx+C \right)  \right) ^{\frac{1}{1-n}}
.\] Calculating the integrating factor, we have: \[
\mu(x) = e^{\int (1-n) P(x) dx} = e^{2 \ln(x+\frac{m}{\rho})} = \left(x+\frac{m}{\rho}\right)^2
.\] Thus we have: \[ 
V(x) = \left( \frac{1}{\left(x+\frac{m}{\rho}\right)^2}\left( 2\int\left( x+\frac{m}{\rho} \right) ^{2}g~dx+C \right)  \right) ^{\frac{1}{2}}
\]\[ 
V(x) = \left( \frac{1}{\left(x+\frac{m}{\rho}\right)^2}\left( \frac{2}{3}\left( x+\frac{m}{\rho} \right) ^{3} +C \right)  \right) ^{\frac{1}{2}} = \frac{1}{x+\frac{m}{\rho}} \sqrt{\frac{2}{3}\left( x+\frac{m}{\rho} \right)^{3}g+C } 
.\]  Plugging in the initial condition, we get: $C=-\frac{2}{3}\frac{m^{3}}{\rho^{3}}g$, giving us: \[
v(x)=  \frac{1}{x+\frac{m}{\rho}}\sqrt{\frac{2}{3}\left( x+\frac{m}{\rho} \right) ^{3}g- \frac{2}{3}\left( \frac{m}{\rho} \right) ^{3}g} 
.\] The acceleration is: \[
g- \frac{v^2}{x+\frac{m}{\rho}}
.\] 

\end{document}

