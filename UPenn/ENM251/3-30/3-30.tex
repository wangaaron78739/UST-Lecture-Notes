\documentclass[../main/main.tex]{subfiles}


\begin{document}

\section{March 30th, 2020}
\subsection{Bessel's ODE and Bessel Functions}
We know that solutions to the ODE: \[
	y''(x) +\lambda^2y(x) = 0
\] are: \[
y_1(x) = \cos(\lambda x)\quad y_2(x) = \sin(\lambda x)
.\] With: \[
y(x) = c_1y_1(x) + c_2y_2(x)
.\] Using the series expansion for sine and cosine, we have: \[ 
y_1(x) = \cos(\lambda x) = \sum_{k=0}^{\infty}  \frac{(-1)^k (\lambda x)^{2k}}{(2k)!}
.\] \[
y_2(x) = \sin(\lambda x) = \sum_{k=0}^{\infty} \frac{(-1)^{k}(\lambda x)^{2k+1}}{(2k+1)!}
.\] 

The Bessel's ODE is the equation: \[
	x^2y''(x) + xy'(x) + (x^2-\nu^2)y(x) = 0, x>0
.\] Without going through the series method, one solution to this ODE is given by: \[
J_\nu (x) = \sum_{k=0}^{\infty} \frac{(-1)^{k}}{k!(k+\nu)!}\left( \frac{x}{2} \right) ^{2k+\nu}, \quad |x|<\infty
.\] We can have any values real of $\nu$ even fractions or irrational values (using the gamma function).
\begin{remark}
	Note that $J_\nu(-x) = (-1)^{\nu}(x)$ and when $\nu$ is an even integer, than $J_\nu(x)$ is an even function, while when $v$ is an odd integer, then $J_\nu(x)$ is an odd funciton.
\end{remark}
\begin{remark}
	If $|x|$ is very small, then $J_\nu(x) \approx \left( \frac{x}{2} \right) ^{\nu}\frac{1}{\nu!}$, $\nu\ge 0$. As such, $J_\nu(x)$ is finite at $x=0$.
\end{remark}
Using Abel's equation, we can get second linearly independent solution to Bessel's ODE, which is: \[
	y_2(x) = Y_\nu(x) = J_\nu(x) = \left( A \int \frac{1}{xJ_\nu^2(x)}~dx+B \right) 
.\] Noteablly, it is not finite at $x=0$.

Thus the general solution to Bessel's ODE is:  \[
	y(x) = c_1J_\nu(x) + c_2Y_\nu(x)
.\] You can always use $Y_\nu(x) $ as the second solution, but if it turns out that $\nu$ is not an integer, we can instead use $J_{-\nu}(x)$ for the second solution. Regardless, $J_{-\nu}(x)$ is also not finite at $x=0$. This means that if we require that $y(\pm 1) $ to be finite, then we must set $c_2=0$, giving us: \[
y(x) = c_1 J_{\nu}(x)
.\] 

There is a \vocab{modified Bessel's ODE}: \[
	x^2y''(x) + xy'(x) - (x^2+\nu^2) y(x) = 0
.\] Which has a general solution \[
y(x) = c_1I_\nu(x) + c_2K_\nu(x)
.\] Where: \[
I_\nu(x) = \sum_{k=0}^{\infty} \frac{1}{k!(k+\nu)!}\left( \frac{x}{2} \right) ^{2k+\nu}
.\] \[
K_\nu(x) = I_\nu(x) \left( A \int \frac{1}{x I^2_\nu(x)} ~dx +B\right) 
.\] 
\begin{remark}
	$I_\nu(x) $ is finite for $\nu\ge 0$, and $K_\nu(x) $ is not finite.
\end{remark}

\subsection{Properties of $J_\nu(x)$ }
\[
	J_{\nu+1}(x) = \frac{2\nu}{x}J_\nu(x) - J_{\nu-1}(x)
.\] \[
	Y_{\nu+1}(x) = \frac{2\nu}{x}Y_\nu(x) - Y_{\nu-1}(x)
 .\] \[ 
	I_{\nu+1}(x) = -\frac{2\nu}{x}Y_\nu(x) + Y_{\nu-1}(x)
 .\]\[ 
	K_{\nu+1}(x) = \frac{2\nu}{x}Y_\nu(x) + Y_{\nu-1}(x)
 .\]  
\begin{theorem}
	Given the ODE: \[
		x^2y''(x) + (a+2bx^{R})xy'(x) + (c+dx^{2s}-b(1-a-R)x^{R}+b^2x^{2R})y(x)=0, x>0
	.\] A general solution is given by: \[
	y(x) = x^{\frac{1-a}{2}}e^{\frac{-bx^{R}}{R}}\left( c_1J_p\left( \frac{\sqrt{d} }{s}x^{s} \right) +c_2 Y_p\left( \frac{\sqrt{d} }{s}x^{s} \right)  \right) 
	.\] when $d>0$, and:  \[
	y(x) = x^{\frac{1-a}{2}}e^{\frac{-bx^{R}}{R}}\left( c_1I_p\left( \frac{\sqrt{-d} }{s}x^{s} \right) +c_2 K_p\left( \frac{\sqrt{-d} }{s}x^{s} \right)  \right) 
	.\] when $d<0$, where: \[
	p = \left| \frac{1}{s}\sqrt{\left( \frac{1-a}{2} \right) ^2-c}  \right| 
	.\] 
\end{theorem}
\begin{example}
	Consider: \[
		xy''(x) + 2y'(x) + \lambda^2x^2y(x) = 0
	.\] Multiplying by $x$, we have: \[
	x^2y''(x_0 + 2xy'(x) + \lambda^2x^{3}y(x) = 0
	.\] Which means that \[a+2bx^{R}=2\implies a=2,b=0.\] We also have: \[
	c+dx^{2s}-b(1-a-R)x^{R}+b^2x^{2R}=\lambda^2x^{3} => c+d^{2s}=\lambda x^{3}
	.\] \[
\to  c=0, d=\lambda^2\ge 0, s = \frac{3}{2}	
	.\] Thus: \[
	p = \left| \frac{1}{s}\sqrt{\left( \frac{1-2}{2} \right)^2-0 }  \right| =\frac{1}{3}
	.\] Thus: \[
	y(x) = x^{-\frac{1}{2}}\left( c_1J_{\frac{1}{3}}\left( \frac{2}{3}\lambda x^{\frac{3}{2}} \right) +c_2 Y_{\frac{1}{3}}\left( \frac{2}{3}\lambda x^{\frac{3}{2}} \right)  \right) 
	.\] 
\end{example}
\end{document}

