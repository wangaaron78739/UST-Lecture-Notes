\documentclass[../main/main.tex]{subfiles}


\begin{document}

\section{March 25th, 2020}
\subsection{Important Result Involving ODEs}
Let us consider the Legendre ODE: \[
	(1-x^2)y''(x) -2xy'(x) + m(m+1) y(x) = 0, \quad -1 < x < 1
.\] Recall that one solution is $y(x) = P_m(x)$. We can generalize this to a class of problems that look like:  \[
a_2(x) y''_m(x) + a_1(x) y_m(x) + a_0(x) y_m(x) + \lambda_m^2b(x) y_m(x) = 0, \quad \alpha < x < \beta
.\] With $\lambda_m$ being some parameter for different values of $m$.
\begin{remark}
	Note that for the Legendre ODE falls into this category with $a_2(x) = 1-x^2$, $a_1(x) = -2x$, $a_0(x) = 0$, $b(x) = 1$, $\lambda_m^2=m(m+1)$
\end{remark}
\begin{theorem}
	If $y_M(x) $ is a solution to the above equation and if $y_N(x)$ is a solution to the same ODE with $M$ replaced by $M$, then after much algebra, we find that: \[
	\left( \lambda_M^2-\lambda_N^2 \right) \int^\beta_\alpha y_M(x)y_N(x)w(x)~dx = \left( s(x)\left| \begin{bmatrix} y_M(x) & y_N(x) \\ y'_M(x) &y_N'(x) \end{bmatrix}\right|  \right) \bigg\rvert_\alpha^\beta
	.\] 
	Where: \[
		s(x) = e^{\int \frac{a_1(x)}{a_2(x)}dx}
	.\] \[
	w(x) = \frac{b(x)}{a_2(x)}s(x)
	.\] 
\end{theorem}
\begin{example}
	Consider $y''_N(x) + N^2y_N(x) = 0, \quad 0<x<\pi$. One solution is $y_N(x) = \sin(Nx)$. Note that this is of the form with $a_2(x) = 1$, $a_1(x) = 0$, $a_0(x) = 0$, $b(x) = 1$, $\lambda_N^2=N^2$. We have \[
		s(x) = e^{\int \frac{0}{1}dx}=1
	.\] \[
	w(x) = \frac{1}{1}(1)=1
	.\] Plugging into the equation, we have: \[
(M^2-N^2)\int^\pi_0 \sin(Mx) \sin(Nx) (1)~dx = \left( (1)\left| \begin{bmatrix} \sin(Mx)&\sin(Nx)\\M\cos(Mx)&N\cos(Nx) \end{bmatrix}  \right|  \right) \bigg\rvert^\pi_0 \]\[= \left| \begin{bmatrix} 0&0\\M(-1)^{M} &N(-1)^{N} \end{bmatrix}  \right| =0
	.\] If $M\neq N$, then: \[
	\int^\pi_0 \sin(Mx) \sin(Nx) ~dx = 0
	.\] 
\end{example}
	For the Legendre ODE, we have: $a_2(x)=1-x^2$, $a_1(x) = -2x$, $a_0=0$, $\alpha_N^2=N(N+1)$, $ b(x)=1$. Then we have: \[
		s(x) = e^{\int \frac{a_1(x)}{a2(x)}dx} = 1-x^2
	.\] \[
	w(x) = \frac{b(x)}{a_2(x)}s(x) = \frac{1}{1-x^2}(1-x^2) = 1
	.\] Using the earlier general result, we have: \[
\left( M(M+1)-N(N+1) \right) \int^-1_-1 P_M(x)P_N(x)(1) ~dx = \left( (1-x^2) \ldots \right)\bigg\rvert^1_{-1} = 0
	.\]
\begin{theorem}
	Thus if $M\neq N$, we have: \[
		\int^1_{-1} P_M(x)P_N(x)~dx = 0
	.\] 
\end{theorem}	
\subsubsection{Legendre Series}
Recall that the Taylor series is a power series expansion: \[
	f(x) = \sum_{n=0}^{\infty} a_n (x-x_0)^{n}, \quad a_n = \frac{f^{(n)}(x_0)}{n!}
.\] 
For this, we are expanding a function in terms of polynomials, with the basis set being: \[
	\{1,(x-x_0),(x-x_0)^2,\ldots\} 
.\] 
The \vocab{Legendre series} is similar, where the basis set being the Legendre polynomials: \[
	\{P_0(x),P_1(x),P_2(x)\ldots\} 
.\] Meaning that: \[
f(x) = \sum_{n=1}^{\infty} a_n P_n(x), \quad -1<x<1
.\] Note that this is an equation with an infinite number of equations and an infinite number of unknowns.
To get the $a_n$, we will make use of the result that: \[
	\int^1_{-1}P_M(x)P_N(x)~dx = 0,\quad M\neq N
.\] Suppose $f(x) $ is given and all the $P_n(x)$ are known. Suppose you want to compute $a_3$. We multiply both sides by $P_3$, giving us: \[
f(x) P_3(x) = \sum_{n=1}^{\infty} a_n P_n(x) P_3(x)
.\] Integrating both sides, we have: \[
\int^1_{-1}f(x) P_3(x) ~dx = \int^1_{-1}\sum_{n=0}^{\infty} a_n P_n(x) P_3(x)~dx
.\] In this situation, we can swap the integral and sum (not always the case, but in this case it turns out to be true), thus we have: \[
= \sum_{n=0}^{\infty} a_n \int^1_{-1}P_n(x) P_3(x)~dx = a_3 \int^1_{-1}P_3(x)P_3(x) ~dx
.\]  
Thus we have: \[
	a_3 = \frac{\int^1_{-1}f(x)P_3(x)~dx}{\int^1_{-1}P_3(x)P_3(x)~dx}
.\] Thus we have: \[
f(x) = \sum_{n=0}^{\infty} a_nP(x)
.\] \[
a_n = \frac{\int^1_{-1}f(x) P_n(x) ~dx}{\int^1_{-1}P_n(x)P_n(x)~dx}
.\] It can be shown that: \[
\int^1_{-1}P_n(x)P_n(x) ~dx = \frac{2}{2n+1}, \quad n=0,1,2,\ldots
.\] To make this easier to remember, remember that if:  \[
A = A_1 i + A_2 j + A_3 k
.\] We have: \[
A_1 = \frac{A\cdot i}{i\cdot i},  A_2 = \frac{A\cdot j}{j\cdot j},  A_3 = \frac{A\cdot k}{k\cdot k}
.\] If we think of $\int f(x)g(x)~dx$ as a "dot product" between equations, we can think of $a_n = \frac{f\cdot P_n}{P_n\cdot P_n}$. Note that this satisfies some special properties that are the same as the dot product between vectors, namely: 
\begin{itemize}
	\item $f\cdot g = g\cdot f$
	\item $f\cdot (g+h) = f\cdot g + f\cdot  h$
	\item $f\cdot\alpha g = \alpha(f\cdot g)$ 
	\item $f\cdot f\ge 0$
	\item $f\cdot f = 0 \iff f=0$
\end{itemize}


\end{document}

