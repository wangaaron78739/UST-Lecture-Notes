\documentclass[../main/main.tex]{subfiles}


\begin{document}

\section{April 13th, 2020}
\subsection{Partial Differential Equation}
\begin{definition}[Wave Equation in 1-Dimension]
	If we have $y(x,t)$ be the position of the point on the string at $x$ and at time $t$, we would have the boundary conditions: \[
	\begin{cases}
		y(\alpha,t) = h_1, \quad 0 \le t \\
		y(\beta,t) = h_1, \quad 0 \le t 
	\end{cases}
	.\] along the left $(x = \alpha)$ and right $(x=\beta)$ boundaries. The equation of motion (which is derived from Newton's 2nd law) is: \[
\begin{cases}
	\frac{\partial }{\partial x} \left( \frac{T(x,t) \frac{\partial y}{\partial x} }{\sqrt{1+\left( \frac{\partial y}{\partial x}  \right) ^2} } \right) = \rho(x) \left( y+ \frac{\partial ^2 y}{\partial t^2}  \right) \sqrt{1+\left( \frac{\partial y}{\partial x}  \right) ^2} \\
\frac{\partial }{\partial x} \left( \frac{T(x,t)}{\sqrt{1+\left( \frac{\partial y}{\partial x}  \right) ^2} } \right) =0
\end{cases} 
.\] Where $T(x,t)$ is the tension function. With initial conditions: \[
y(t,x) \bigg\rvert_{t=0} = y(x,0) = y(x) \quad \text{(starting position)}
.\] \[
\frac{\partial y(x,t)}{\partial t} \bigg\rvert_{t=0}= v(x)\quad \text{(starting velocity)}
.\] Some simplifying assumptions are: \[
T(x,t) = T(x) \quad \text{ meaning that the tension only depends on $x$}
.\]  \[
\frac{\partial y}{\partial x}  \ll 1 \quad \text{ meaning that the tension is so high, it's hardly moving}
.\] 
Under these assumptions, we have $\sqrt{1+\left( \frac{\partial y}{\partial x}  \right) ^2}=1 $ , making the PDE: \[
\begin{cases}
	T_0 \frac{\partial ^2 y}{\partial x^2}  = \rho (x) \left( g + \frac{\partial ^2 y }{\partial t^2}  \right) \\
	\frac{\partial }{\partial x} \left( T(x) \right)  = 0 \implies T(x) = T_0 \text{ which is a constant}
\end{cases}
.\] If we further assume that $\rho(x)= \text{constant}$, we would get: \[
T_0 \frac{\partial ^2 y}{\partial x^2} = \rho_0 g + \rho_0 \frac{\partial ^2 y}{\partial t^2} 
.\] 
Thus this problem becomes given $T_0, \rho_0, h_1, h_2, y(x), v(x)$, solve for $y(x,t)$ if: \[
T_0 \frac{\partial ^2 y}{\partial x^2}  = \rho_0 g + \rho_0 \frac{\partial ^2 x}{\partial t^2} , \quad \alpha < x < \beta, 0 < t
.\] \[
y(\alpha,t) = h_1 \quad y(\beta,t) = h_2, \quad 0 \le  t
.\] \[
y(x,0) = y(x) , \quad \alpha < x < \beta
.\] \[
\frac{\partial y(x,t)}{\partial t} \bigg\rvert_{t=0} = v(x) , \quad \alpha < x < \beta
.\] This is called a boundary-value, initial-value problem (BCIVP). We can shift it by setting $\alpha = 0$ and $\beta = L$.
\end{definition}
\begin{remark}
	BVIVP = PDE + BC + IC
\end{remark}
To solve we will do the following: 
\begin{enumerate}
	\item First ask the question are the PDE and boundary conditions Homogeneous? i.e. does $y(x,t) = 0$ satisfy the PDE and boundary conditions. If the answer is no, proceed to step 2, if yes proceed to step 6.
	\item Construct the ODE and BCs satisfied by the time independent solution $y(x,t) = y_e(x)$ (equilibrium solution). To construct, in the PDE, replace $y(x,t)$ by $y_e(x)$. For the 1-D wave equation, we have: \[
			T_0 \frac{\partial ^2 y_e(x)}{\partial x^2} = \rho_0 g + \rho_0 \underbrace{\frac{\partial ^2 y_e(x)}{\partial t^2} }_{=0} 
	\implies T_0 y_e''(x) = \rho_0 g, \quad 0 < x < L
	.\] For the BCs, we have: \[
	\begin{cases}
		y_e(0) = h_1 \\ 
		y_e(L) = h_2
	\end{cases}
	.\] 
\item Solve the ODE and BCs for $y_e(x)$. For the 1-D wave equation, we have: \[
		y_e''(x) = \frac{\rho_0g}{T_0} \implies y_e'(x) = \frac{\rho_0g}{T_0}x + c_1 \implies y_e(x) = \frac{\rho_0 gx^2}{2T_0} + c_1 x + c_2
.\]  \[
y_e(0) = c_2 = h_1 \implies y_e(x) = \frac{\rho_0g x^2}{2T_0}+c_1 x + h_1
.\] \[
y_e (L) = \frac{\rho_0gL^2}{2T_0}+c_1L + h_1 = h_2 \implies c_1 = \frac{h_2-h_1}{L}- \frac{\rho_0gL}{2T_0}
.\] Giving us: \[
y_e(x) = \frac{\rho_0gx^2}{2T_0}+ \left( \frac{h_2-h_1}{L}-\frac{\rho_0gL}{2T_0} \right) x + h_1 , \quad 0 \le  x \le  L
.\] This is known as the \vocab{time-independent (steady state, equilibrium) shape}.
\item Define  \[
		u(x,t) = y(x,t) -y_e(x)
	.\] This can represent the shifting of the string from the equilibrium shape.
\item Construct the BVIVP satisfied by $u(x,t)$. This can done by substituting: \[
		y(x,t) = u(x,t) + y_e(x)
		,\] into the BVIVP for $y(x,t)$. For the 1-D wave equation, we would get: \[
		T_0 \frac{\partial^2 (u + y_e)}{\partial x^2} = \rho_0 g + \rho_0 \frac{\partial ^2\left( u+y_e \right) }{\partial t^2}  
.\] \[
\implies T_0 \frac{\partial ^2 u}{\partial x^2} + \underbrace{T_0 y_e''(x)}_{\rho_0 g} = \rho_0 g + \rho_0 \frac{\partial ^2 u}{\partial t^2} 
.\] \[
\implies T_0 \frac{\partial ^2 u}{\partial x^2}  = \rho_0 \frac{\partial ^2 u}{\partial t^2} , \quad 0 < x < L , x < t
.\] For the BC, we have: \[
y(0,t) = h_1 \implies u(0,t) + \underbrace{y_e(0)}_{=h_1} = h_1 \implies u(0,t) = 0
.\] and \[
y(L,t) = u(L,t) + \underbrace{y_e(L)}_{=h_2} = h_2 \implies u(L,t) = 0
.\] For the IC, we have: \[
y(x,0) = y(x) \implies u(x,0) = y(x) -y_e(x)
.\]  \[
\frac{\partial (u+y_e)}{\partial t}\bigg\rvert_{t=0} \implies \frac{\partial u}{\partial t} \bigg\rvert_{t=0} = v(x) 
.\] Overall this means that: \[
\text{PDE} \begin{cases}
	T_0 \frac{\partial ^2 u(x,t)}{\partial x^2} = \rho_0 \frac{\partial ^2 u(x,t)}{\partial t^2} ,\quad 0 < x < L , 0 < t
\end{cases}
.\] \[
\text{BCs} \begin{cases}
	u(0,t) = 0 \\ u(L,t) = 0
\end{cases}, \quad 0 < t
.\] \[
\text{ICs} \begin{cases}
u(x,0) = y(x) - y_e(x) \\ \frac{\partial u(x,t)}{\partial t} \bigg\rvert_{t=0} = v(x)
\end{cases}, \quad 0 < x < L
.\] Note that now the PDE and BCs are now homogeneous. We will continue from here next lecture.
\end{enumerate}
\end{document}

