\documentclass[../main/main.tex]{subfiles}


\begin{document}

\section{April  1st, 2020}
\subsection{Dot Products}
Recall that we defined the generic dot product between functions to be: \[
	f\cdot g = \int^\beta_\alpha f(x)g(x) w(x) ~dx
.\] With $w(x) $ being a weight function. Recall that: \[
\vec{A}\cdot \vec{B} = |\vec{A}||\vec{B}| \cos \theta ,\quad \cos\theta = \frac{\vec{A}\cdot \vec{B}}{\sqrt{\vec{A}\cdot \vec{A}} \sqrt{\vec{B}\cdot \vec{B}}  } 
.\] 
For \[
	f\cdot g = \int^\beta_\alpha f(x)g(x) w(x) ~dx 
.\] we have: \[
|f| = \sqrt{f\cdot f} 
.\] \[
\cos \theta = \frac{f\cdot g}{\sqrt{f\cdot f} \sqrt{g\cdot g} }= \frac{f\cdot g}{|f| |g|}
.\] 
Note that $f$ is perpendicular to $g$ is  $f\cdot g = 0$.
\subsection{Recall Bessel's Equation}
 Consider \[
	 x^2y''(x)+(a+2b^{R})xy'(x) + (c+dx^{2s}-b(1-a-R)x^{R}+b^2x^{2R})y(x) =0
.\] 
\begin{example}[Example where Bessel's equation does not work]
	\[
		y''(x) + y'(x) + x^{3}y(x) = 0
	.\] Multiplying by $x^2$, we have: \[
	x^2y''(x) + y'(x) + x^{5}(x) =0
	.\] This means that $a+2bx^{R}=x$, meaning that: \[
	a=0, b=\frac{1}{2}, R=1
	.\] \[
	c+dx^{2s}-b(1-a-R)x^{R}+b^2x^{2R}=x^{5}
	.\] \[
	c+dx^{2s}+\frac{1}{4}x^2=x^{5}
	.\] \[
	c+dx^{2s}=x^{5-\frac{1}{4} x^{2}}
	.\] Which does not work because $c$ and $d$ are constants.
\end{example}
\begin{example}
	\[
		y''(x) + x^{5}y(x) = 0
	.\] Multiplying by $x^2$, we have: \[
	x^2y''(x) + x^{5}y(x) = 0
	.\] \[
	\implies a+2bx^{R}=0 \implies a=0, b=0
	.\] \[
	c+dx^{2s}-b(1-a-R)x^{R}+b^2x^{2R}=x^{5}\implies c=0, d=1>0, s = \frac{5}{2}
	.\] \[
	p = \left| \frac{1}{s} \sqrt{\left( \frac{1-a}{2} \right) ^2-c}  \right|  = \frac{1}{5}\neq  \text{ integer}
	.\] \[
	y(x) = x^{\frac{1-a}{2}} e^{-b\frac{x^{R}}{12}}\left( c_1J_p\left( \frac{\sqrt{d} }{s}x^{s} \right)+c_2 J_p\left( \frac{\sqrt{d} }{s}x^{5} \right)   \right) 
	.\] 
	\[
		y(x) = x^{1/2}\left( c_1 J_{\frac{1}{5}} )_\frac{2}{5}x^{\frac{5}{2}}+c_2 J_{\frac{1}{5}} \left( \frac{2}{5}x^{\frac{5}{2}} \right) \right) 
	.\] For small $x$, $J_\mu(x) \approx \frac{1}{\nu!}x^{\nu}\to  x^{\nu}$. Thus for small $x$ \[
	y(x) = c_1 x+c_2
.\] Thus if: \[
y(x) = 0 \implies c_2 = 0
.\] Giving us: \[
y(x) = x^{1/2} c_1J_{1/5}\left( \frac{2}{5}x^{5/2} \right) 
.\] 
\end{example}
Ging back to vectors: \[
	\vec{A} = a\vec{i} + b\vec{j} + c\vec{k}
.\] $\{\vec{i},\vec{j},\vec{j}\} $ is called an orthogonal basis set. Similarly you can have an infinite dimensional basis set: \[
\{ \phi_1(x),\phi_2(x),\phi_3(x),\ldots\}  
.\] if $\phi_m\cdot \phi_n=0$ for $m\neq n$ with respect to some inner product

\end{document}

