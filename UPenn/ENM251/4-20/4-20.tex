\documentclass[../main/main.tex]{subfiles}


\begin{document}

\section{April 20th, 2020}
\subsection{Heat Equation}
If we have a one dimensional rod with ends $x=\alpha$ and $x=\beta$, with
\begin{itemize}
	\item mass density: $\rho(x)$
	\item conductivity: $\kappa(x)$
	\item specific heat per unit mass: $c(x)$
	\item temperature at point $x$ at time $t$: $u(x,t)$
	\item heat generation term within rod: $R(x,t)$ (for our purpose, assume  $R(x,t)=0$ )
\end{itemize}
Using conservation of energy, we get: \[
	\frac{\partial }{\partial x} \left( \kappa(x) \frac{\partial u}{\partial x}  \right) = \rho(x) c(x) \frac{\partial u}{\partial t} 
.\] This is known as the \vocab{one dimensional heat equation}. 
\begin{remark}
	If we compare this with the one dimensional wave equation from earlier: \[
		\frac{\partial }{\partial x} \left( T(x) \frac{\partial u}{\partial x}  \right) = \rho(x) \frac{\partial ^2 u}{\partial t^2} 
	.\] The only difference is that the wave equation has a second order partial derivative, while the heat has a first. 
	
	This makes a difference, as the second order derivatives means that the wave equation has oscillations, while the heat equation has exponential decay.
\end{remark}
Let us consider the special case where $\kappa(x) = \kappa_0$, $\rho(x) = \rho_0$, $c(x) = c_0$, which gives us: \[
\kappa_0 \frac{\partial ^2 u}{\partial x^2}  = \rho_0c_0 \frac{\partial u}{\partial t} \implies \frac{\partial ^2 u}{\partial x^2} = \frac{1}{\gamma} \frac{\partial u}{\partial t} ,\quad \gamma= \frac{k_0}{\rho_0c_0} 
.\] Note that since we can shift this equation to $x=0$ and $x=L$.
\begin{remark}
	$\gamma_0$ is often known as the diffusity constant.
\end{remark}
Now for the boundary conditions, let us fix $x=0$ to be a certain temperature $T_1$, while we expose $x=L$ to a room of temperature $T_2$. This gives us the boundary conditions: \[
u(0,t) = T_1\]\[ h u(L,t) + \kappa_0 \frac{\partial u(x,t)}{\partial x} \bigg\rvert_{x=L} = hT_2
.\] 
\begin{remark}
	Note that these boundary conditions are only examples, if we fixed the temperature at $x=L$, we would have $u(L,t) = T_2$.
\end{remark}
Now for the initial condition, note that since the order of the derivative is only one, we only need one initial condition: \[
	u(x,0) = f(x), 0 < x < L
.\] Thus the mathematical formulation of the problem is:

Given the constants, $\rho_0, c_0, \kappa_0, h, T_1, T_2, \gamma= \frac{\kappa_0}{\rho_0c_0}$ and the function $f(x)$, solve for  $u(x,t)$ in the region $x\le x \le  L$,  $0\le t$, if: \[
	\text{PDE}\begin{cases}
		\frac{\partial ^2 u(x,t)}{\partial x^2}  = \frac{1}{\gamma} \frac{\partial u(x,t)}{\partial t} ,\quad 0 < x < L , \quad 0 < t 
	\end{cases}.\] \[
	\text{BCs}\begin{cases}
		u(0,t) = T_1, \quad 0 < t\\
hu(L,t) + \kappa_0 \frac{\partial u(x,t)}{\partial x} \bigg\rvert_{x=L} = hT_0, \quad 0 < t
	\end{cases}
.\]
\[
	\text{ICs}\begin{cases}
		
u(x,0) = f(x) , \quad 0 < x < L
	\end{cases}
.\] 

Just like the wave equation, we ask the following questions: 
\begin{enumerate}
	\item Is the PDE and BCs Homogeneous by setting $u(x,t) = 0$
		\begin{itemize}
			\item PDE: yes
			\item BC ($x=0$): no
			\item BC ($x=L$): no
		\end{itemize}
	If at least one of these is no, we move to step 2, otherwise we can skip to step 5. 
\item Define and construct the boundary value problem satisfied by some time independent solution $u_e(x)$: \[
		\frac{\partial ^2 u_e(x)}{\partial x^2}  = \frac{1}{\gamma} \frac{\partial u_e(x)}{\partial t} 
.\] \[
\implies u_e''(x) = 0, \quad 0 < x < L
.\] \[
u_e(0) = T_1 \quad hu_e(L) + \kappa_0 u_e'(L) = hT_2
.\] 
\item Solve this BVP for $u_e$. For this case: \[
		u_e(x) = Ax + B
.\] \[
u_e(0) = B = T_1 \implies u_e'(x) = A
.\] \[
h u_e(L) + \kappa_0 u_e'(x) = h T_2 \implies h(AL+T_1) + \kappa_0 A = hT_2
.\] \[
\implies A = \frac{h(T_2-T_1)}{hL+\kappa_0}
.\] Thus the time independent solution is: \[
u_e(x) = \frac{h(T_2-T_1)}{hL+\kappa_0}x + T_1, \quad 0 \le  x \le  L
.\] 
\item Define: \[
		v(x,t) = u(x,t) - u_e(x)
	.\] and the construct the BVIVP satisfied by $v(x,t)$: \[
	\frac{\partial ^2(v(x,t)+u_e(x)}{\partial x^2} = \frac{1}{\gamma}\frac{\partial (v(x,t)+u_e(x)}{\partial t} 
	.\] \[
	\implies \frac{\partial ^2 v(x,t)}{\partial x^2} +\underbrace{u_e''(x)}_{=0} = \frac{1}{\gamma} \left( \frac{\partial v(x,t)}{\partial t} +0 \right) 
	.\] \[
	\implies \frac{\partial ^2 v(x,t)}{\partial x^2} = \frac{1}{\gamma}\frac{\partial v(x,t)}{\partial t} , \quad 0 < x < L, \quad 0 < t
	.\]\[
	v(0,t) + u_e(0) = T_1 \implies v(0,t) = 0, \quad 0 < t
	.\] \[
h u(L,t) + \kappa_0 \frac{\partial u(x,t)}{\partial x} \bigg\rvert_{x=L} = hT_2 \implies h v(L,t) + \kappa_0 \frac{\partial v(x,t)}{\partial x}  = 0, \quad 0 < t
	.\] \[
	v(x,0) = f(x) - u_e (x) , \quad 0 < x < L
	.\] Note that the PDE and BCs for this BVIVP are homogeneous. 
\item Construct a general solution to the PDE and BCs for $v(x,t) = u(x,t) - u_e(x)$ by constructing a basis set of solutions of the form:  \[
		v(x,t) = \phi(x) \beta(t) \neq 0
.\] Plugging in the PDE, we get: \[
\frac{\partial ^2 \left( \phi(x)\beta(t) \right) }{\partial x^2} = \frac{1}{\gamma} \frac{\partial \left( \phi(x)\beta(t) \right) }{\partial t}  
.\] \[
\implies \phi''(x) \beta(t) = \frac{1}{\gamma}\phi(x) \beta'(t)
.\] \[
\implies \frac{\phi''(x)}{\phi(x) }= \frac{1}{\gamma} \frac{\beta'(t)}{\beta(t)} = \text{ a constant }C
.\] This gives us: \[
\phi''(x) - V \phi(x) = 0, \quad 0 < x < L
.\]  \[
\beta'(t) - \gamma c \beta(t) = 0 , \quad 0 < t
.\]with boundary conditions \[
v(0,t) = 0 \implies \phi(0) = 0
.\] \[
h \phi(L) + \kappa_0 \phi'(L) = 0
.\] Note that this is a RSLP for $x$.

Recall the this has solutions: \[
\phi(x) = \begin{cases}
	A \cosh (x \sqrt{C} ) + B \sinh(x\sqrt{C} ) , \quad C > 0\\
	A+Bx, \quad C = 0 \\
	A \cos (x\sqrt{-C} ) + B \sin(x \sqrt{-C} ) , \quad C< 0
\end{cases}
.\] As with before, note that for the cases $C>0$ and $C=0$, we would get $\phi(x) = 0$, with the only case $C<0$ : \[
\phi(x) = A \cos(x\sqrt{-C} ) + B \sin(x \sqrt{-C} ) 
.\] \[
\phi(0) = A = 0 \implies \phi(x) = B \sin(x \sqrt{-C} ) \implies \phi'(x) = B \sqrt{-C} \cos(x \sqrt{-C} )
.\] \[
h \phi(L) + \kappa_0 \phi'(L) = 0 \implies B\left( h \sin (L\sqrt{C} ) + \kappa_0\sqrt{-C} \cos(L\sqrt{-C} ) \right) =0
.\] \[
\implies \tan \left( L \sqrt{-C}  \right) = - \frac{\kappa_0}{h}\sqrt{-C}  
.\] Once again, this will give us $\lambda_1, \lambda_2, \ldots$ which are the eigenvalues for this problem. We will need to solve this numerically.

Once we have these, we have: \[
	\phi_n(x) = \sin(\lambda_n x ), \quad n = 1,2,3,\ldots
.\] We will continue from this point next lecture.
\end{enumerate}
\end{document}

