\documentclass[../main/main.tex]{subfiles}


\begin{document}

\section{February 16th, 2021}
\subsection{Basic Iterative Method}
In this chapter, we will introduce iterative methods. There will be a lot of overlap with MATH5311. For iterative methods, we make use of the fact that matrix vector products are fast for sparse matrices.

\begin{remark}
If the matrix is sparse, then the matrix vector product is on the order of non-zero entries.
\end{remark}
\begin{example}
For the Discrete Laplacian in 2D, the matrix vector product is $O(N)$.
\end{example}
We will solve $Ax = b$ by stationary iterative methods. Given $x_{k}\in\RR^{n}$, we want to improve the quality of $x_{k}$ using: \[
x_{k+1} = Gx_{k} + f, \quad k\in 0,1,2 \ldots
\] where $G\in\RR^{n\times n}$ and $f\in\RR^{n}$ are stationary matrices and vectors.
\begin{definition}\index{stationary matrix}
 $G$ is a \vocab{stationary matrix}, as it does not depend on $k$.
\end{definition}
\subsection{Jacobi Iteration}
\begin{itemize}
  \item $(y)_{i}$ denotes the $i$-th component of a vector $y$
        \item $\xi_{i}^{(k)}$ denotes the $i$-th component of $x_{k}$
        \item $\xi_{i}$ denotes the $i$-th component of $x$ (true solution)
        \item $\xi_{i}$ denotes the $i$-th component of $b$
\end{itemize}
The idea of the Jacobi iteration is, given $x_{k}$, we obtain $x_{k+1} $ by solving the $i$-th unknown from the $i$-th equation.
More precisely, we are solving: \[
(Ax-b)_{i} = 0,
\] with $\xi_{j}$, $j\neq i$, fixed to be $\xi_{j}^{(k)}$, for $i=1, \ldots , n$. As such, we have:
\begin{align*}
  &(Ax-b)_{i} = 0 \\
  &\iff a_{ii} \xi_{i} ^{(k+1)} + \sum_{j\neq i} a_{ij} \xi_{j}^{(k) } = \beta_{i} \\
    & \iff \xi_{i}^{(k+1)} = (\beta_{i} - \sum_{j\neq i} a_{ij} \xi_{j}^{(k) }) / a_{ii}
  .\end{align*}
This can be expressed as Algorithm \ref{2-16-jac1}.
        \begin{algorithm}[h!]
	\caption{Element Wise Jacobi Iteration}
    \label{2-16-jac1}
	\begin{algorithmic}[1]
      \For{ $k=0,1,2, \ldots$}
      \For{ $i=1, \ldots , n$}
      \State $\xi_{i}^{(k+1)} = (\beta_{i} - \sum_{j\neq i} a_{ij} \xi_{j}^{(k) }) / a_{ii}$
      \EndFor
      \EndFor
	\end{algorithmic}
	\end{algorithm}
    In order to perform this effeciently, we reformulate this in matrix notation to make use of BLAS. Let $A=D-E-F$, where:
    \[
A = \begin{bmatrix} d_1 &  & *\\ & \ddots & \\ * & & d_n \end{bmatrix} =  \begin{bmatrix} d_1 &  & 0\\ & \ddots & \\ 0 & & d_n \end{bmatrix}  -  \begin{bmatrix} 0 &  & 0\\ & \ddots & \\ -* & & 0 \end{bmatrix}  -  \begin{bmatrix} 0 &  & -*\\ & \ddots & \\ 0 & & 0 \end{bmatrix}  = D-E-F
    \]
    Thus, we have Algorithm \ref{2-16-jac2}, which is in stationary form.
        \begin{algorithm}[h!]
	\caption{Jacobi Iteration in Matrix Form}
    \label{2-16-jac2}
	\begin{algorithmic}[1]
      \For{ $k=0,1,2, \ldots$}
      \State $x_{k+1} = D^{-1}(b+(E+F)x_{k})$
      \EndFor
	\end{algorithmic}
	\end{algorithm}
    \begin{remark}
      Some other equivalent forms of the Jacobi Iteration are: \[
x_{k+1} = D^{-1}(E+F)x_{k} + D^{-1} b
\] \[
x_{k+1} = D^{-1}(D-A)x_{k} + D^{-1} b
\] \[
x_{k+1} = (I-D^{-1} A) x_{k} + D^{-1} b
\]
    \end{remark}
\subsection{Review on Norms}
\subsubsection{Vector Norms}
\index{norm}
\begin{definition}
  A (vector) \vocab{norm} on $R^{n}$ is a function $\|\cdot\| : \RR^{n}\to \RR$ that satisfies:
  \begin{enumerate}
    \item $\| x\| \geq 0\quad \forall x \in\RR^{n}$ and $\|x\|= 0 \iff x = 0$.
    \item $\|\alpha x\|= |\alpha| \|x\| \quad \forall \alpha \in\RR$ and $x \in\RR^{n}$.
          \item $\|x+y\| \leq \|x\|+ \|y\| \quad \forall x,y \in\RR^{n}$ (\vocab{triangle inequality}).
  \end{enumerate}
  This defines a \vocab{metric } on $\RR^{n}$.
\end{definition}
\begin{definition}\index{$p$ norm}
  A $p$-norm on $\RR^{n}$ is defined as: \[
          \|x\|_{p} = \left(\sum_{i=1}^n|x_{1}|^{p}\right)^{1 / p}
  \]
\end{definition}

\begin{example}[Spedial $p$ Norm]
  Here are a few common norms on $\RR^{n}$.
  \begin{itemize}
    \item \vocab{ $p$-norm } $(p\geq1)$: \[
          \]
    \item \vocab{Euclidean norm} ($p=2$) \[
          \|x\|_{2} = \left(\sum_{i=1}^n|x_{1}|^{2}\right)^{1 / 2}
          \]
    \item \vocab{1-norm} ($p=1$) \[
          \|x\|_{1} = \sum_{i=1}^n|x_{1}|
          \]
    \item \vocab{$\infty$-norm} ($p=\infty$) \[
          \|x\|_{\infty} = \max_{i=1}^n|x_{1}|
          \]
  \end{itemize}
\end{example}
\begin{theorem}[\vocab{Holder's Inequality}] \[
|x^{T} y| \leq \|x\|_{p} \|y\|_{q}
  \] if $\frac{1}{p} + \frac{1}{q} = 1$, with $p,q \geq 1$.
  \index{Holder's inequality}
\end{theorem}

\begin{theorem}[\vocab{Cauchy-Schwartz Inequality}]
  \[
|\langle u,v\rangle| \leq \| u\| \|v\|,\quad \forall u,v \in\RR^{n} \]
  \index{Cauchy-Schwartz inequality}
\end{theorem}
\begin{example}[Weighted Norm]
  Let $A\in\RR^{n\times n}$ be a symmetric positive definite matrix. Then: \[
\|x\|_{A} = (x^{T} A x)^{1 / 2}
  \] is a norm, called the \vocab{weighted norm}. \index{weighted norm}
\end{example}

From functional analysis, because $\RR^{n}$ is finite dimensional, any two norms are equivalent. More formally.
\begin{theorem}[Norm equivalence of $\RR^{n}$]
  Given $\|\cdot\|_{a}$ and $\|\cdot\|_{b}$, $\exists C_1,C_2 >0$ independent of $x$, s.t. \[
    C_1\|x\|_{b} \leq \|x\|_{a} \leq C_2 \|b\| \forall x \in\RR^{n}
  \]
  \label{2-16-conv}
\end{theorem}
Consequently, from Theorem \ref{2-16-conv}, the convergence of vectors in $\RR^{n}$ under any norm is the same. Thus, we can analyze the convergence under any norm.

\begin{remark}
Theorem \ref{2-16-conv} does not hold for infinte dimensional space. However, for numerical analysis, we always work with finite dimensional space.
\end{remark}
\begin{example}[Equivalence of $1$-norm and other $p$-norms]
  \[
    \|x\|_{2}\leq \|x\|_{1} \leq \sqrt{n} \|x\|_{2} \quad \forall x\in\RR^{n}
  \] \[
    \|x\|_{\infty}\leq \|x\|_{1} \leq n \|x\|_{\infty} \quad \forall x\in\RR^{n}
  \]
\end{example}
\subsubsection{Matrix Norm}
Let $\|\cdot\|$ be a norm on $\RR^{n}$. Let $A\in\RR^{n \times n}$ be a matrix.
\begin{definition}\index{matrix norm}
  \label{2-16-matnorm}
  The \vocab{norm of $A$ induced by the vector norm $\| \cdot\|$} is: \[
\|A\| = \sup_{x\in\RR^{n}, x\neq 0}\frac{\|Ax\|}{\|x\|} = \max_{\|x\|=1}\|Ax\|
  \]
\end{definition}
\begin{remark}
The second equality in \ref{2-16-matnorm} is due to the scaling property of $A$ and because the norm is continuous. However, this might not be the case in infinite-dimensional spaces.
\end{remark}
We can check that $\|A\|$ is a matrix, i.e.:
\begin{itemize}
    \item $\| A\| \geq 0\quad \forall A \in\RR^{n \times n}$ and $\|A\|= 0 \iff A = 0$.
    \item $\|\alpha A\|= |\alpha| \|A\| \quad \forall \alpha \in\RR$ and $A \in\RR^{n \times n}$.
          \item $\|A+B\| \leq \|A\|+ \|B\| \quad \forall A,B\in\RR^{n\times n}$ (\vocab{triangle inequality}).
\end{itemize}
In addition, since it is an \vocab{operator norm} that is induced, it has some consistency properties, namely
\begin{itemize}
\item $\|AB\| \leq \|A\| \|B\|\quad \forall A,B \in\RR^{n\times n}$
        \item $\|Ax\|\leq \|A\|\|x\|\quad \forall A \in\RR^{n\times n}, x\in\RR^{n}$
\end{itemize}
\begin{example}[matrix 2-norm] \[
\| A\|_{2} = \max_{\|x\|_{2}=1} \|Ax\|_{2} = \left(\max_{\|x\|_{x=1}}\|Ax\|_{2}^{2}\right)^{ \frac{1}{2} } = \left(\max_{x^{T}x=1}x^{T}A^{T}Ax\right)^{\frac{1}{2}}
\] \[
 = (\text{maximum eivenvale of }A^{T}A)^{\frac{1}{2}}
\]which is the maximum \vocab{singular value} of $A$.
  \label{2-16-sin}
\end{example}
\begin{remark}
The last equality in Example \ref{2-16-sin} can be shown by taking the eigenvalue decomposition of $A$.
\end{remark}
\begin{theorem}
  \[
    \|A\|_{1}= \max_{1\leq j \leq n}\|a_{j}\|_{1}, \quad\text{where }A = \begin{bmatrix}
a_1 &a_2&\ldots&a_{n}
    \end{bmatrix} , a_{j}\in\RR^{n}
 , \] i.e. the maximum column 1-norm (column sum).
\end{theorem}
\begin{proof}
\begin{itemize}
  \item $\forall x \in\RR^{n}$ with $\|\|_{1 = 1}$, we have: \[
        \|Ax\|_{1} = \|\sum_{j=1}^n x_{j}a_{j}\|_{1} \leq \sum_{j=1}^n |x_{j}| \|a_{j}\|_{1} \leq \max_{1\leq j\leq n} \|a_{j}\|_{1} \sum_{j=1}^n |x_{j}| = \max_{1\leq j\leq n} \|a_{j}\|_{1}
        \]
        Taking the max over all $x: \|x\|_{1}=1$, we obtain: \[
\|A\|_{1} \leq \max_{1\leq j\leq n} \|a_{j}\|_{1}
        \]
  \item Let $j_{0} = \arg\max_{1\leq j\leq n} \|a_{j}\|_{1}$. Consider $x=e_{j_{0}}$. Then $\|x\|_{1}=1$ and $Ax=Ae_{j_{0}}= a_{j_{0}}$. Thus: \[
\|Ax\|_{1}= \|a_{j_{0}}\|_{1} = \max_{1\leq j\leq n} \|a_{j}\|_{1}
        \] Therefore: \[
        \|A\|_{1}\geq \|Ax\|_{1} =\max_{1\leq j\leq n} \|a_{j}\|_{1} \]
\end{itemize}
\end{proof}
\begin{remark}
This means that for the matrix 1-norm, the maximum is attained at the image of one of the standard unit vector. This is true, since the 1-ball is a convex polytope.
\end{remark}
\begin{theorem}
  \[
    \|A\|_{\infty}= \max_{1\leq i \leq n}\|a^{(i)}\|_{\infty}, \quad\text{where }A = \begin{bmatrix}
(a^{(1)})^{T}\\
\vdots
(a^{(n)})^{T}\\
    \end{bmatrix} , a^{(i)}\in\RR^{n}
 , \] i.e. the maximum row 1-norm (maximum row sum).
\end{theorem}
\begin{proof}
  (omitted).
\end{proof}
\begin{definition}\index{spectral radius}
  The \vocab{spectral radius} of a matrix $A$ is defined as: \[
    \rho(A) = \max \{|\lambda_{i}| : \lambda_{i} \text{ is an eigenvalue of }A
    \}\]
\end{definition}
\begin{theorem}
  Let $A\in\CC^{n\times n}$. Then:
  \begin{enumerate}
\item $\|A\|\geq \rho(A)$ for any matrix norm induced by $\|\cdot\|$.
    \item For any $\epsilon>0$, we can find a vector norm $\|\cdot\|$, s.t. the induced matrix norm satsfies: \[
          \|A\|\leq \rho(A)+\epsilon
          \]
          \item
  From (1) and (2), we have: \[
          \rho(A) = \inf \|A\|
          \]
    \item If $A$ is diagonalizable, there exists a matrix operator norm s.t. \[
          \rho(A) = \|A\|
          \]
          \item In particular, when $A$ is symmetric, $\rho(A) = \|A\|_{2}$.
  \end{enumerate}
\end{theorem}
\begin{proof}
  \begin{enumerate}
    \item Let $\lambda_{0},x_{0}$ be an eigenpair of $A$ satisfying $|\lambda_{0}| = \rho(A)$. Assume that $\|x_0\|=1$. Then, for any vector norm $\|\cdot\|$, its induced operator norm satisfies: \[
          \|A\|\geq \|Ax_{0}\| = \|\lambda_{0}x\| = |\lambda_{0}| \|x_{0}\| = \rho(A)
          \]
    \item We use a construction proof by finding such vector norm. Let \[
A = X \begin{bmatrix}
      \lambda_{1}& \delta_{1}&&& &\\
      & \lambda_{2} & \delta_{2}&&& \\
      & & \ddots & \ddots &&\\
      & &  & \lambda_{n-1} &\delta_{n-1}&\\
      &&&&& \lambda_{n}  \\
    \end{bmatrix}
          \] be the Jordan decomposition, where: $\delta_{i} \in \{ 0,1\}$, and $\lambda_{i}$ are eigenvalues of $A$.

          Given $\epsilon > 0$, we define: \[
          \|x\|_{\epsilon} = \|(X D_{\epsilon})^{-1} x\|_{\infty}, \quad \text{with }D_{\epsilon} = \begin{bmatrix}
1&&&& \\
&\epsilon&&& \\
&&\epsilon^2&& \\
&&&\ddots& \\
&&&&\epsilon^{n-1} \\
          \end{bmatrix}.
          \] We can check that $\|\cdot\|_{\epsilon}$ is a norm on $\CC^{n}$. So: \[
\|A\|_{\epsilon} = \max_{\|x\|_{\epsilon}=1} \|Ax\|_{\rho} = \max_{\|(XD_{\epsilon})^{-1} x\|_{\infty}=1} \|(XD_{\epsilon})^{-1}Ax\|_{\infty}
          \]Let $y=(XD_{\epsilon})^{-1} x$, we have: \[
          \|A\|_{\epsilon} = \max_{\|y\|_{\infty}=1} \|(XD_{\epsilon})^{-1} A (XD_{\epsilon})y\|_{\infty} = \|(XD_{\epsilon})^{-1}A(XD_{\epsilon})\|_{\infty}
          \] Note that we have: \[
          (XD_{\epsilon})^{-1} A (XD_{\epsilon}) = D_{\epsilon}^{-1} X^{-1} A X D_{\epsilon} = D_{\epsilon} \begin{bmatrix}
      \lambda_{1}& \delta_{1}&&& \\
      & \lambda_{2} & \delta_{2}&& \\
      & & \ddots & \ddots &&\\
      & &  & \lambda_{n-1} &\delta_{n-1}\\
      &&&& \lambda_{n}  \\
          \end{bmatrix}D_{\epsilon}
          \] \[
          = \begin{bmatrix}
      \lambda_{1}&\epsilon \delta_{1}&&& \\
      & \lambda_{2} &\epsilon \delta_{2}&& \\
      & & \ddots & \ddots &&\\
      & &  & \lambda_{n-1} &\epsilon\delta_{n-1}\\
      &&&& \lambda_{n}  \\
          \end{bmatrix}
          \]
          Thus, since the infinity norm is the maximum row sum, we have: \[
          \|A\|_{\epsilon} \leq \max_{1\leq i\leq n} (|\lambda_{i}| +\epsilon) \leq \rho(A) +\epsilon
          \]
          \item By part 2, if $A$ is diagonalizable, $\delta_{i} = 0$ for all $i$. Then $\|A\|_{\epsilon} = \max_{i} |\lambda_{i}| = \rho(A)$.
          If $A = A^{T}$, then $\delta_{i} = 0$ for all $i$, $\lambda_{i}$ are real, and $X$ is unitary. Thus: \[
          A^{T}A = X^{-1} \begin{bmatrix}
\lambda_{1}^2&&& \\
&\lambda_{2}^2&& \\
&&\ddots& \\
&&&a\lambda_{n}^2 \\

          \end{bmatrix}
          \]
          \[
          \rho(A) = (\rho(A^{T}A))^{\frac{1}{2}} = \|A\|_{2}
          \]
  \end{enumerate}

\end{proof}
\begin{remark}
  1. and 2. imply \[
\rho(A) = \inf \{\|A\|: \|\cdot\| \text{ is an operator norm }\}
  \]
  In particular
  \begin{itemize}
    \item If $A$ diagonalizable, then minimum is attainable, meaning:  \[
\rho(A) = \min \{\|A\|: \|\cdot\| \text{ is an operator norm }\}
          \]
    \item If $A$ is symmetric: \[
\rho(A) = \|A\|_{2} = \min \{\|A\|: \|\cdot\| \text{ is an operator norm }\}
          \]
  \end{itemize}
\end{remark}


\end{document}
