\documentclass[../main/main.tex]{subfiles}


\begin{document}

\section{February 25th, 2021}
\subsection{Gauss-Seidel Iteration}\index{Gauss-Seidel Iteration}
Recall that in Jacobi, $\xi_{i}^{(k+1)}$ are updated parallely in Algorithm \ref{2-16-jac1}.

\vocab{Gauss-Seidel} is just modifying Jacobi to be updated ``successively''. When we modify $\xi_{i}^{(k+1)}$ we can use $\xi_{1}^{(k+1)} ,\ldots \xi_{i-1}^{(k+1)}$. This gives us Algorithm \ref{2-25-ga}.
        \begin{algorithm}[h!]
	\caption{Gauss-Seidel Iteration}
    \label{2-25-ga}
	\begin{algorithmic}[1]
      \For{ $k=0,1,2, \ldots$}
      \For{ $i=1, \ldots , n$}
      \State $\xi_{i}^{(k+1)} = (\beta_{i} - \sum_{j> i} a_{ij} \xi_{j}^{(k) } - \sum_{j< i} a_{ij} \xi_{j}^{(k+1) } ) / a_{ii}$
      \EndFor
      \EndFor
	\end{algorithmic}
	\end{algorithm}

    In the matrix reformulation, recalling that $A=D-E-F$, where $E$ is lower triangular and $F$ is upper triangular, we have: \[
    \]
    \begin{align*}
     & x_{k+1} = D^{-1}(b+F x_{k}+E x_{k+1}) \\
      &\iff (D-E)x_{k+1} = b +F x_{k} \\
   & \iff x_{k+1} = (D-E)^{-1} F x_{k} + (D-E)^{-1} b \\
    &  \iff x_{k+1} = \underbrace{(I - (D-E)^{-1} A)}_{G} x_{x} + \underbrace{(D-E)^{-1}}_{f} b
                     .\end{align*}
                   which is in the stationary iteration form.\\

                   \begin{remark}
In the algorithm above, we are updating $\xi_{i}^{(k+1)}$ in regular numerical order (from $1$ to $n$). However, we can use other ordering, such as going from $n$ to $1$ in reverse order. Gauss-Seidel is sensitive to the ordering of updating of unknowns.
                   \end{remark}
\subsubsection{Convergence of Gauss-Seidel}
Since Gauss-Seidel is a stationary iteration, we can use the same framework as the Jacobi iteration, with: \[
(x_{k}-x_{*}) = G^{k}(x_{0}-x_{*})
\]
\begin{theorem}
Gauss-Seidel converges to $x_{*}$ for any $x_{0}$ if and only if $\rho(G)<1$.
\end{theorem}
% \begin{proof}
% Same as the proof for Jacobi.
% \end{proof}
The convergence speed is $\|G\|$, as we have: \[
\|x_{k+1}-x_{*}\| \leq \|G\|\|x_{k}-x_{*}\|
\] In particular, we have:  \[
\|G\| = \begin{cases}
  \rho(G) & \text{ if $G$ is symmetric and $\|\cdot\|=\|\cdot\|_{2}$ }\\
  \rho(G)+\epsilon & \text{otherwise}
\end{cases}
\]with $\epsilon$ arbitrarily small.
\begin{theorem}
Let $A \in\RR^{n \times n}$ be SPD. Then $\rho(G)< 1$.
\end{theorem}
\begin{proof}
  Recall $G=I-(D-E)^{-1} A$. Let $(\lambda,u)$ be an eigenpair of $G$ (note that $\lambda,u$ may be complex). We have:
  \begin{align*}
Gu =  \lambda u \\
    (I-(D-E)^{-1}A u) = \lambda u \\
    (D-E)^{-1} F u = \lambda u \\
    Fu = \lambda (D-E) u
    .\end{align*}
  Since $A$ is symmetric, we have: \[
A=A^{T} \implies F = E^{T}
.\] Thus, we have:
\begin{align*}
  \lambda(D-E) u &= E^{T} u \\
\lambda u^{*} (D-E)u&= u^{*}E^{T} u
                                 .\end{align*}
                               by  left-multiplying both side by $u^{*}=\overline{u}^{T}$  \\

                               Let us set $u^{*} E u = \alpha +i \beta \in \CC$ and $u^{*}D u = \delta \in \RR$, since $D$ is real and symmetric. Then, we have:
                               \[
u^{*}E^{T} u = (u^{*}E^{T} u)^{T} = u^{T} E (u^{*})^{T} = \overline{u^{*}} \overline{E} \overline{u} = \overline{u^{*}E u} = \alpha - i \beta
\]
Since $A$ is SPD,  $D$ is also SPD, giving us $u^{*}D u > 0$. \\

Thus  \[
\lambda(\delta - (\alpha+i\beta)) = \alpha-i\beta \implies \lambda = \frac{\alpha-i\beta}{(\delta-\alpha)-i\beta}
\]Giving us: \[
|\lambda|^2 = \frac{(\alpha^2+\beta^2)}{(\delta-\alpha)^2+\beta^2}
\]

Because: \[
(\delta-\alpha)^2+\beta^2 = \delta^2+\alpha^2+\beta^2 - 2\alpha\delta = (\alpha^2+\beta^2)+\underbrace{\delta}_{>0}(\delta-2\alpha)
\] we need to check the sign of $\delta-2\alpha$. \\

Since $A $ is SPD, $u^{*}Au>0 \implies u^{*}(D-E-E^{T})u = \delta -(\alpha+i\beta) - (\alpha - i\beta) = \delta -2\alpha > 0 $. As such: \[
(\delta - \alpha)^2 +\beta^2 > \alpha^2 +\beta^2 \implies |\lambda|^2 < 1
\]
\end{proof}

\begin{remark}
This is more general than Jacobi, which is convergent if and only if $2D-A$ is SPD.
\end{remark}
\begin{example}
  Recall that the 1D discrete Laplacian is SPD, meaning that Gauss-Seidel converges. Similarly for the 2D case.
\end{example}
\begin{example}
  Gauss-Seidel converges for irriducible diagonally dominant matrices. This is used for laplacian on irregular domains.
\end{example}
\subsection{Acceleration of G-S / and Jacobi (SOR)}\index{SOR iteration}

Here we try to accelerate Gauss-Seidel and Jacobi by considering momentum acceleration. We can consider the iteration of $x_{k}$ to $x_{k+1}$ as a movement by the Gauss-Seidel iteration. The idea of SOR is to consider the momentum of the movement until some stage. \\

For Gauss-Seidel, the components are:
\begin{itemize}
\item  $x_{k}:$ $ \xi_{i}^{(k)}$
\item  $x_{k+1}:$ $ \xi_{i}^{(k+1)} = (\beta_{i} - \sum_{j> i} a_{ij} \xi_{j}^{(k) } - \sum_{j< i} a_{ij} \xi_{j}^{(k+1) } ) / a_{ii}$
\end{itemize}
To consider the momentum, we would have:
  \begin{align*}
 \xi_{i}^{(k+1)} &= \xi_{i}^{(k)} +\omega (\xi_{i}^{(k+1)}- \xi_{i}^{(k)})\\
  &= \xi_{i}^{(k)} +\omega \left((\beta_{i} - \sum_{j> i} a_{ij} \xi_{j}^{(k) } - \sum_{j< i} a_{ij} \xi_{j}^{(k+1) } ) / a_{ii}- \xi_{i}^{(k)}\right)
    .\end{align*} for some $\omega > 1$, giving us the Algorithm \ref{2-25-sor}.

        \begin{algorithm}[h!]
	\caption{SOR Iteration}
    \label{2-25-sor}
	\begin{algorithmic}[1]
      \For{ $k=0,1,2, \ldots$}
      \For{ $i=1, \ldots , n$}
      \State $\xi_{i}^{(k+1)} = \xi_{i}^{(k)} +\omega \left[(\beta_{i} - \sum_{j> i} a_{ij} \xi_{j}^{(k) } - \sum_{j< i} a_{ij} \xi_{j}^{(k+1) } ) / a_{ii}- \xi_{i}^{(k)}\right]$
      \EndFor
      \EndFor
	\end{algorithmic}
	\end{algorithm}

    In the matrix reformulation, we have: \[
x_{k+1} = x_{k} + \omega \left[D^{-1}(b+F x_{k}+ E x_{k+1})- x_{k}\right]
\] Expressing this in its stationary form, we have: \[
x_{k+1} = \underbrace{(D-\omega E)^{-1} \left((1-\omega)D+\omega F\right)}_{G_{\omega}} x_{k} + \underbrace{\omega (D-\omega E)^{-1}}_{f_{\omega}} b
\]
Now the question is finding a suitable $\omega$.

\subsubsection{Convergence of SOR}

\begin{theorem}
If $\omega \not\in(0,2)$, then $\rho(G_{\omega})>1$.
\end{theorem}
\begin{proof}
  We have:
  \[
    \det(G_{\omega}) = \det((D-\omega E)^{-1}) \cdot \det((1-\omega)D +\omega F)
  \]
\begin{itemize}
        \item
  Since $D-\omega E$ is lower triangular, $(D-\omega E)^{-1}$ is also lower triangular. Thus, $\det((D-\omega E)^{-1})$ is the product of its diagonals, which are the inverse of the diagonal of $D-\omega E$ which are $\frac{1}{a_{i i}}$. As such, we have: \[
    \det ((D-\omega E)^{-1}) = \prod_{i=1}^n a_{ii}^{-1}
        \]
  \item Since $(1-\omega)D +\omega F$ is upper triangular, we have: \[
        \det ((1-\omega)D +\omega F) = \prod_{i=1}^n (1-\omega)~a_{ii}
        \]
\end{itemize}
Thus, we have: \[
  \det (G_{\omega}) = \prod_{i=1}^n (1-\omega)~a_{ii} \cdot a_{ii}^{-1} = (1-\omega)^{n} = \prod_{i=1}^n \lambda_{i}
\]
This gives us:
\begin{align*}
  & |1-\omega|^{n} = | \lambda_{1}|| \lambda_{2}|\ldots| \lambda_{n}| \leq (\rho(G_{\omega}))^{n} \\
  \implies & |1-\omega| \leq \rho(G_{\omega})
  .\end{align*}
\end{proof}
\begin{corollary}
In order for SOR to converge, we must have $\omega < 2$.
\end{corollary}
\begin{itemize}
\item If $\omega = 1$, we have Gauss Seidel
  \item If $\omega  \in (0,1)$, we have successive under relaxation
        \item If $\omega \in (1,2)$ we have successive over relaxation (SOR).
\end{itemize}
\begin{theorem}
If $A$ is SPD, then $\rho (G_{\omega })<1$ for all $\omega \in (0,2)$.
\end{theorem}
\begin{proof}
Similar to Gauss-Seidel.
\end{proof}
How do we pick the optimal $\omega $? Roughly speaking, with the optimal $\omega $:
\begin{itemize}
\item $\rho (G_{\omega_{\text{opt}}})\sim \rho (G_{1})^2$. Where $G_{1}$ is the iteration matrix in G-S.
        \begin{remark}
This means that SOR is roughly two times faster than G-S, since it is the square.
        \end{remark}
        \item $\rho (G_{1})\sim (\rho(G_{\text{Jacobi}}))^2$.
\end{itemize}

\end{document}
