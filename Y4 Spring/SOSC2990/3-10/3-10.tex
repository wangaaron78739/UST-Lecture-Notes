\documentclass[../main/main.tex]{subfiles}


\begin{document}

\section{March 10th, 2021}
\subsection{Moral Development}
\begin{definition}[\vocab{Moral Development}]\index{moral development}
  The changes in people's sense of justice. The process of learning to distinguish between right and wrong in accordance with cultural values.
\end{definition}
\begin{remark}
A person's moral development is closely tied to their cultural background and values.
\end{remark}
There are three dimensions of moral development:
\begin{itemize}
\item Moral Behavior - behavioral perspective
\item Moral Emotions - psychodynamic perspective
\item Moral Reasoning - cognitive perspective
\end{itemize}
Moral development does not exist at birth, but as babies are exposed to socialization, they begin to learn what is write and wrong.
\subsection{Moral Behavior}\index{moral behavior}
\begin{itemize}
\item Focuses on how the environment in which the individual operates produces moral behavior.
  \item According to the operant conditioning theory, it is the \textbf{consequences} of behaviors that teach children to obey moral rules.
        \index{opearnt conditioning theory}
    \item Moral behaviors (e.g. prosocial behaviors) are reinforced, while morally unacceptable behaviors are punished.
\end{itemize}

\begin{definition}\index{prosocial behavior}
\vocab{Prosocial behaviors} are any behaviors that can help or benefit another person.
\end{definition}
\begin{example}
Mia helping / hitting her brother will be reinforced or punished by the parent.
\end{example}
Although punishment is effective, it shouldn't be overdone.
For effective parenting, we should always follow the punishment with an explanation, or tell them about alternate behaviors.\\

\begin{itemize}
        \item Apart from operant conditioning theory, there is the observational learning theory to learn what's right and wrong.  \index{observational learning theory}
\item Children learn more from role models, and rewarded behaviors will be imitated while punished behaviors will be avoided.
        \item Research supports this theory, as children learn moral behavior from a model's behaviors.
        \item We don't require direct punishment to learn morally acceptable behavior.
\end{itemize}
\begin{example}
Children were told a story about a person who was misbehaving and got punished. Children in the experimental condition were more behaved.
\end{example}
\end{document}
