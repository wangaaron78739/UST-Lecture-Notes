\documentclass[../main/main.tex]{subfiles}


\begin{document}

\section{February 24th, 2021}
\subsection{Operant Conditioning - Skinner Box}\index{operant conditioning}
Apart from classical conditioning theory, another behavioral theory is operant conditioning theory, discovered by Skinner while studying in ``Skinner boxes''.
\begin{example}[\vocab{Skinner Box Experiment}]\vocab{Skinner box}
\begin{itemize}
  \item The rat is placed in a box with a lever. When the rat presses the lever, it is given food.
  \item At the beginning, the rat does not know that when they press the lever, they are given food, but after a while
\end{itemize}
\end{example}
\begin{definition}[\vocab{Reinforcement}]\index{reinforcement}
If the behavior is followed by positive consequences, the behavior becomes more likely.
\end{definition}
\begin{definition}[\vocab{Punishment}]\index{punishment}
If the behavior is followed by negative consequences, the behavior becomes less likely.
\end{definition}

\begin{remark}
There are both positive and negative reinforcement and punishment.
\end{remark}
\begin{description}
  \item[Positive Reinforcement:]  A behavior is followed by a pleasant stimulus.
        \begin{example}
Giving a raise for good performance.
        \end{example}
  \item[Negative Reinforcement:]  A behavior is followed by the removal of an unpleasant stimulus.
        \begin{example}
          Applying ointment to a itchy rash.
        \end{example}
  \item[Positive Punishment:]  A behavior is followed by a unpleasant stimulus.
        \begin{example}
          Yelling at a child for stealing.
        \end{example}
  \item[Negative Punishment:]  A behavior is followed by the removal of an pleasant stimulus.
        \begin{example}
          Getting grounded for misbehaving.
        \end{example}
\end{description}
\subsubsection{Discussion}
\begin{quote}
  ``Is reinforcement or punishment more effective to discipline children?''
\end{quote}
\begin{example}
I think it depends on how well the children can take stress and adversity. If the child can take stress and adversity, punishment might take an opposite effect then expected. It could also be depending on how the child perceive reinforcement/punishment vs how the parents think.
\end{example}
\begin{example}
I think reinforcement is more effective to encourage children to do something and punishment is more effective when prevent them from doing something.
\end{example}

\begin{example}
I guess both are needed. Since children may not be mature enough to understand all the rules that their parents set, so both punishment and reinforcement could let the children learn if certain behavior is socially acceptable. However, the parents should explain the reason behind the rules to have a long term effect.
\end{example}

\subsection{Bandura's Social Cognitive Theory}
Bandura's \vocab{social cognitive theory} says that:
\begin{itemize}
\item Behavior can be learned without direct experience
        \begin{example}
You don't have to be hit by a car to know that it is dangerous.
        \end{example}
\item Studying learning in terms of thought processes that underlie it.
\end{itemize}
\begin{definition}[\vocab{Observational Learning}]
  A process in which an individual learns new responses by observing what others do and what happens to them for doing it, instead do through direct experience.
\end{definition}
\begin{remark}
  Imitation of role models is more likely when:
  \begin{itemize}
    \item The role model is prestigious, smart, popular, or talented
    \item Target behavior is desired by the society
          \item The role model is similar to you
  \end{itemize}
\end{remark}
\begin{remark}
Learn chunks of behaviors and integrate them into new, complex behavioral pattern.
\end{remark}
\subsubsection{Bobo Doll Experiment} \index{bobo doll experiment}
\begin{itemize}
  \item Children are exposed to two conditions, in one, the child is shown an adult being aggressive to it, and one where the adult is not.
    \item After watching, the child is put in a room with a bobo doll with toy hammers.
        \item For the children who watch the adult being aggressive, the children also acted aggressive towards it.
        \item Those in the control group were less likely to act aggressively towards the bobo doll.
\end{itemize}
\begin{remark}
The children who are exposed to adults that are acting violently model their action. Thus, learning can be done through observations.
\end{remark}


\subsection{Evaluation of Behavioral Perspectives}
\subsubsection{Contributions}
\begin{itemize}
  \item Evidence do support that behaviors are developed by conditioning and modeling
        \item Particularly useful in explaining emotional responses
\end{itemize}

\subsubsection{Limitations}
\begin{itemize}
  \item Too optimistic that behaviors can be changed by changing the environment
        \begin{example}
Raising the price of cigarettes does little to affect smoking.
        \end{example}
  \item Humans are not passive recipients of environmental influences. People will change/affect their
        \begin{example}
People will change which role model they will follow. Even if we are exposed to a selected role model, it is not guaranteed that they will model themselves off of them.
        \end{example}

    \item Does not explain age-related changes, e.g. how children think differently from adults.
\end{itemize}


\subsection{Cognitive Theories}

Cognitive theories broadly states that \textbf{``Development is a process of age-related changes in thinking and reasoning''}. We will look at the following theories in this course:
\begin{itemize}
\item Piaget's Cognitive Theory
\item Information Processing Theory
\item Vygotsky's Sociocultural Theory
\item Cognitive Neuroscience Approach
\end{itemize}
We will first look at the latter two in this lecture, and return to Piaget and Information Processing theory later.
\subsection{Vygotsky's Sociocultural Theory}\index{Vygotsky's sociocultural theory}
According to Vygotsky's Sociocultural Theory, complex forms of thinking have their origin in social interactions. Cognitive development is the result of social interaction, and thus are different for different cultures/society.
\begin{example}
  Children's toys reflect difference in cultural values. In western cultures, toy cars are more mechanical
  In some African cultures, there are also cars, but they are often hand-crafted.
\end{example}
Children's acquisition of cognitive skills is guided by a more skilled person.

\begin{remark}
In laymen's terms, Vygotsky believes that
\end{remark}
Vygotsky proposed two ideas:
\begin{itemize}
  \item Zone of proximal development
        \item Scaffolding
\end{itemize}
\begin{definition} \index{zone of proximal development}
The \vocab{zone of proximal development (ZPD)} is the distance between what the child can do by themselves and the next learning that they can be helped to achieve with competent assistance.
\end{definition}
\begin{remark}
If the child is provided with assistance (appropriate scaffolding), they are able to achieve substantially more than they can on their own.
\end{remark}
\begin{definition}\index{scaffolding}
\vocab{Scaffolding} is an instructional approach in which a more knowledgeable other provides scaffolds or supports to facilitate the learner's development.
\end{definition}
\begin{remark}
Often, there are tasks that are too difficult for the child to perform on their own. However, with proper scaffolding, these tasks might become manageable.
\end{remark}
\begin{example}
Parents helping children with proper motor skill, e.g. cutting shapes in paper.
\end{example}
As time goes by, the amount of scaffolding decreases, as they take on more responsibility.
\begin{remark}
The main idea with scaffolding is that we have to decrease the amount of support over time to instill the sense of \vocab{autonomy} in the child.
\end{remark}
\begin{example}
Scaffolding can be applied to cooking:
\begin{itemize}
\item First, the child might only be mixing ingredients
  \item Later on, they might be to cut ingredients
    \item Eventually, the child can cook on their own
\end{itemize}
\end{example}
\subsection{Discussion}
\begin{quote}
``According to Vygotsky's theory, what kind of education programs would be effective in promoting a child's development?''
\end{quote}
\begin{example}
Guided case-base learning with debriefing may help, with guided case-base learning, information/materials could be broken down into smaller pieces for learning. While debriefing could also help child to understand what they couldn’t do and how to improve with support/guidance.
\end{example}
\begin{example}
More in class activities, where teachers can give immediate feedback and help to the kids
\end{example}
\begin{example}
  other than spoon feeding method, I think we can just let the student to find the answer first instead of give the answer directly, I think of the maths lessons as example, maybe some simple theory is taught then give a simple question to students to solve it by themselves first , then tell them what is the answer , and the difficulties of questions increase gradually with the step above
\end{example}
\subsection{Summary of Discussion}
\begin{itemize}
\item
Teachers should know the abilities of the students. This is important to know, in order to know what the child is able to achieve on their own, and what kind of assistance should be offered.
        \item
Activities should be challenging enough to the children so that they can maximize their abilities and expand their limits.
  \item Instead of spoon-feeding the students, the students should undergo assisted discovery, e.g. the math example
\end{itemize}
\subsection{Evaluation of Cognitive Theories}

\subsubsection{Contributions}
\begin{itemize}
        \item A better account of the internal processes that shape behavior
\end{itemize}

\subsubsection{Limitations}
\begin{itemize}
  \item Ignores the huge variations between people in how the think and act
        \item Neglect the influence of emotions, as our emotions will affect how we think or behave.
\end{itemize}



\subsection{Cognitive Neuroscience Approach}

The cognitive neuroscience approach examines cognitive development through the lens of brain processes an d neurological activity. This is a recent field of development.





\end{document}
